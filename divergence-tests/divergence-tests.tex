\documentclass{ximera}

\title{Divergence and integral tests}

\begin{document}

\begin{abstract}
  If the limit of the terms of a series isn't zero, then the series diverges.
\end{abstract}

\maketitle

\section{Why does $\sum n/(n+1)$ diverge?}

\youtube{https://www.youtube.com/watch?v=HvQrSigs8bA}

\begin{question}
  Does the series \(\displaystyle\sum_{n=2}^\infty \left( \displaystyle\frac{{\left(7 \, n + 8\right)} {\left(7 \, n + 4\right)}}{{\left(4 \, n + 3\right)} {\left(3 \, n + 1\right)}} \right)\) converge or diverge?

  \begin{solution}
    \begin{hint}
      The first thing we should do when confronted with an unknown series is check the limit of the terms.
    \end{hint}
    \begin{hint}
      Recall that if the series converges, then the limit must exist and be equal to zero.
    \end{hint}
    \begin{hint}
      So if the limit does not exist or is not zero, then the series diverges.
    \end{hint}
    \begin{hint}
      In this case, we consider \(\displaystyle\lim_{n \to \infty} \displaystyle\frac{{\left(7 \, n + 8\right)} {\left(7 \, n + 4\right)}}{{\left(4 \, n + 3\right)} {\left(3 \, n + 1\right)}}\).
    \end{hint}
    \begin{hint}
      Let's multiply by a &ldquo;disguised version of one,&rdquo; meaning \(\displaystyle\lim_{n \to \infty} \displaystyle\frac{{\left(7 \, n + 8\right)} {\left(7 \, n + 4\right)}}{{\left(4 \, n + 3\right)} {\left(3 \, n + 1\right)}} \cdot \displaystyle\frac{1/n^{2}}{1/n^{2}}\).
    \end{hint}
    \begin{hint}
      Expanding the numerator and denominator yields \(\displaystyle\lim_{n \to \infty} \displaystyle\frac{\displaystyle\frac{49 \, n^{2} + 84 \, n + 32}{n^{2}}}{\displaystyle\frac{12 \, n^{2} + 13 \, n + 3}{n^{2}}}\).
    \end{hint}
    \begin{hint}
      Dividing through by \(n^{2}\) gives \(\displaystyle\lim_{n \to \infty} \displaystyle\frac{\displaystyle\frac{84}{n} + \displaystyle\frac{32}{n^{2}} + 49}{\displaystyle\frac{13}{n} + \displaystyle\frac{3}{n^{2}} + 12}\).
    \end{hint}
    \begin{hint}
      Note then that \(\displaystyle\lim_{n \to \infty} \displaystyle\frac{\displaystyle\frac{84}{n} + \displaystyle\frac{32}{n^{2}} + 49}{\displaystyle\frac{13}{n} + \displaystyle\frac{3}{n^{2}} + 12} = \displaystyle\frac{49}{12}\).
    \end{hint}
    \begin{hint}
      But \(\displaystyle\frac{49}{12} \neq 0\), and so, this series must diverge.
      
    \end{hint}
    
    
    \begin{multiple-choice}
      \choice[correct]{The series diverges.}
      \choice{The series converges.}
      
    \end{multiple-choice}
    
  \end{solution}
\end{question}

\begin{question}
  Does the series \(\displaystyle\sum_{n=5}^\infty \left( -6 \, \left(\displaystyle\frac{10}{9}\right)^{n} \right)\) converge or diverge?

  \begin{solution}
    \begin{hint}
      This is a geometric series.
    \end{hint}
    \begin{hint}
      Note that the common ratio between subsequent terms is \(r = \displaystyle\frac{10}{9}\).
    \end{hint}
    \begin{hint}
      Note that \(|r| \geq 1\).
    \end{hint}
    \begin{hint}
      Consequently, the series diverges.
      
    \end{hint}
    

    \begin{multiple-choice}
      \choice[correct]{The series diverges.}
      \choice{The series converges.}
    \end{multiple-choice}

  \end{solution}
\end{question}

\section{Does $\sum 1/n$ converge or diverge?}

%Watch \href{https://www.youtube.com/watch?v=_Ui5_-lIK34}{a video about the harmonic series}.

% Relevant video: harmonic-series
\begin{question}
  Does the series \(\displaystyle\sum_{j=0}^\infty \left( \displaystyle\frac{8}{7 \, j + 42} \right)\) converge or diverge?
  
  \begin{solution}
    \begin{hint}
      This series resembles the harmonic series \(\displaystyle\displaystyle\sum_{n=1}^\infty \displaystyle\displaystyle\frac{1}{n}\).
    \end{hint}
    \begin{hint}
      Recall that \(\displaystyle\displaystyle\sum_{n=1}^\infty (c \, a_n) = c \displaystyle\displaystyle\sum_{n=1}^\infty a_n\) when \(\displaystyle\displaystyle\sum_{n=1}^\infty a_n\) converges.
    \end{hint}
    \begin{hint}
      When \(\displaystyle\displaystyle\sum_{n=1}^\infty a_n\) diverges, so does \(\displaystyle\displaystyle\sum_{n=1}^\infty (c \, a_n)\) provided \(c \neq 0\).
    \end{hint}
    \begin{hint}
      Let's use \(c = \displaystyle\frac{7}{8}\) in this case.
    \end{hint}
    \begin{hint}
      Then the series \(\displaystyle\sum_{j=0}^\infty \left( \displaystyle\frac{8}{7 \, {\left(j + 6\right)}} \right)\) converges exactly when \(\displaystyle\sum_{j=0}^\infty \displaystyle\frac{7}{8} \left( \displaystyle\frac{8}{7 \, {\left(j + 6\right)}} \right)\) converges.
    \end{hint}
    \begin{hint}
      Simplifying, \(\displaystyle\sum_{j=0}^\infty \left( \displaystyle\frac{8}{7 \, {\left(j + 6\right)}} \right)\) converges exactly when \(\displaystyle\sum_{j=0}^\infty \left( \displaystyle\frac{1}{j + 6} \right)\) converges.
    \end{hint}
    \begin{hint}
      But the terms of \(\displaystyle\sum_{j=0}^\infty \left( \displaystyle\frac{1}{j + 6} \right)\) are just the tail of the divergent harmonic series!
    \end{hint}
    \begin{hint}
      So the series \(\displaystyle\sum_{j=0}^\infty \left( \displaystyle\frac{8}{7 \, {\left(j + 6\right)}} \right)\) diverges.
      
    \end{hint}
    
    \begin{multiple-choice}
      \choice[correct]{The series diverges.}
      \choice{The series converges.}
      
    \end{multiple-choice}
    
  \end{solution}
\end{question}

\section{How can integrating help us to address convergence?}

\youtube{https://www.youtube.com/watch?v=JkRL4s80KrU}

\section{How can integrating help analyze $\sum 1/n$?}

%\youtube{https://www.youtube.com/watch?v=TODKCEjmN_4}

\end{document}
