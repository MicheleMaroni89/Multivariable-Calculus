\documentclass{ximera}

\title{Absolute convergence}

\begin{document}

\begin{abstract}
  To consider absolute convergence of a series is to consider the convergence of the series where each term is replaced by its absolute value.
\end{abstract}

\maketitle

\section{What is absolute convergence?}

Our goal is to consider the convergence of \(\sum_{n=1}^\infty a_n\) when some of the terms in the sequence \((a_n)\) are positive, and some are negative.  One way to simplify this is to first consider \(\sum_{n=1}^\infty |a_n|\) which is a series in which all the terms are nonnegative---exactly the sort of series we have had a ton of experience with.  If \(\sum_{n=1}^\infty |a_n|\) converges, we say that the original series \(\sum_{n=1}^\infty a_n\) converges absolutely.

\youtube{https://www.youtube.com/watch?v=eE1xSAgiHvU}

\begin{theorem}
Absolute convergence implies convergence.
\end{theorem}
This means that if \(\sum_{n=1}^\infty |a_n|\) converges, then we can conclude that the more difficult to study series \(\sum_{n=1}^\infty a_n\) converges as well. 

\youtube{https://www.youtube.com/watch?v=uZVZ7chhqqg}

\begin{warning}
The converse is not true!
\end{warning}

Just because \(\sum_{n=1}^\infty |a_n|\) diverges does not mean \(\sum_{n=1}^\infty a_n\) necessarily diverges; this situation is referred to as conditional convergence.

\youtube{https://www.youtube.com/watch?v=OGzsP1cUKkA}

% Relevant video: absolute-convergence
            \begin{question}
              Suppose \((a_n)\) is a sequence involving both positive and negative numbers, and suppose that the series \(\displaystyle\sum_{n=1}^\infty |a_n|\) converges.  What can be known for certain about the series \(\displaystyle\sum_{n=1}^\infty a_n\)?
              \begin{solution}
                \begin{hint}
                  We are told that \(\displaystyle\sum_{n=1}^\infty |a_n|\) converges, which means that the series is absolutely convergent.
                \end{hint}
                \begin{hint}
                  Since \(\displaystyle\sum_{n=1}^\infty |a_n|\) converges, so too does \(\displaystyle\sum_{n=1}^\infty 2|a_n|\).
                \end{hint}
                \begin{hint}
                  But note that \(0 \leq a_n+|a_n| \leq 2|a_n|\).
                \end{hint}
                \begin{hint}
                  Consequently, by the comparison test, the series \(\displaystyle\sum_{n=1}^\infty \left( a_n + |a_n| \right)\) converges, since it has nonnegative terms and is smaller termwise than the convergent series \(\displaystyle\sum_{n=1}^\infty 2|a_n|\).
                \end{hint}
                \begin{hint}
                  Consider \(\displaystyle\sum_{n=1}^\infty (a_n+|a_n|) -\displaystyle\sum_{n=1}^\infty |a_n|\).
                \end{hint}
                \begin{hint}
                  This is a difference of convergent series.
                \end{hint}
                \begin{hint}
                  So \(\displaystyle\sum_{n=1}^\infty (a_n+|a_n|) -\displaystyle\sum_{n=1}^\infty |a_n| = \displaystyle\sum_{n=1}^\infty \left( a_n+|a_n|-|a_n| \right)\) converges.
                \end{hint}
                \begin{hint}
                  But that is just to say that \(\displaystyle\sum_{n=1}^\infty \left( a_n+|a_n|-|a_n| \right) = \displaystyle\sum_{n=1}^\infty a_n \) converges.
                \end{hint}
                \begin{hint}
                  So we conclude that the series \(\displaystyle\sum_{n=1}^\infty a_n \) converges.
                \end{hint}
                \begin{hint}
                  This is usually stated as a theorem: absolute convergence implies convergence.

                \end{hint}


              \begin{multiple-choice}
                \choice[correct]{The series \(\displaystyle\sum_{n=1}^\infty a_n\) converges.}
                \choice{The series \(\displaystyle\sum_{n=1}^\infty a_n\) diverges.}
                \choice{The series \(\displaystyle\sum_{n=1}^\infty a_n\) converges conditionally.}
                \choice{The series \(\displaystyle\sum_{n=1}^\infty a_n\) may converge or diverge.}
                \choice{The series \(\displaystyle\sum_{n=1}^\infty a_n\) converges to a positive value.}

              \end{multiple-choice}

              \end{solution}
            \end{question}

\textbf{tl;dr:} You should check for conditional convergence by first checking absolute convergence, and then if you find the series is not absolutely convergent, you can consider convergence of the original series.

\youtube{https://www.youtube.com/watch?v=t4Qm3sb4zBA}

\section{Conditional convergence is terrible.}

A key point to emphasize is just how terrible conditional convergence really is. \href{ http://en.wikipedia.org/wiki/Riemann_series_theorem}{If you rearrange the terms in a conditionally convergence theorem, you can get any value you desire.}

The alternating harmonic series \(\sum_{n=1}^\infty \frac{(-1)^{n+1}}{n}\) is a context within which to consider this rearrangement theorem.

\youtube{https://www.youtube.com/watch?v=7oWMaNg4Ezw}

The upshot is that a series \textbf{must be added up in order} since the answer---in the case of conditional convergence---depends very much on the order in which the addition is performed.  \textit{Be afraid!}

\end{document}
