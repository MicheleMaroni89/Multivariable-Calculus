\documentclass{ximera}

\title{What are power series?}

\begin{document}

\begin{abstract}
  A power series is, very roughly speaking, a polynomial that goes on and on.
\end{abstract}

\maketitle

Up until now, we had been considering series one at a time; with power
series, we are considering a whole family of series,
namely \[\sum_{n=0}^\infty a_n x^n\] which depend on a parameter
\(x\).  They are like polynomials, so they are easy to work with.  And
yet, lots of functions we care about---like \(e^x\)---can be
represented as power series, so power series bring the relaxed
atmosphere of polynomials to the harder realm of functions like
\(e^x\).

\begin{question}
  Which of the following is a power series?

  \begin{solution}
    \begin{hint}
      Remember the form of a power series is \(\displaystyle\sum_{n=0}^\infty a_n x^n\).
    \end{hint}
    \begin{hint}
      So if we separate out the \(x^n\) but find that the remaining \(a_n\) has an \(x\) in it, we are in trouble!
    \end{hint}
    \begin{hint}
      Is \(\displaystyle\sum_{n=0}^\infty \sin\left(x^{n}\right)\) a power series?
    \end{hint}
    \begin{hint}
      No.  There is no \(x^n\) term.  The only \(x^n\) is inside \(\sin\).
    \end{hint}
    \begin{hint}
      Is \(\displaystyle\sum_{n=0}^\infty x^{-n}\) a power series?
    \end{hint}
    \begin{hint}
      No.  The power of \(x\) is negative.
    \end{hint}
    \begin{hint}
      Is \(\displaystyle\sum_{n=0}^\infty \tan\left(x^{n}\right)\) a power series?
    \end{hint}
    \begin{hint}
      No.  There is no \(x^n\) term.  The only \(x^n\) is inside \(\tan\).
    \end{hint}
    \begin{hint}
      How about \(\displaystyle\sum_{n=0}^\infty \displaystyle\frac{2}{63} \, x^{n} {\left(14 \, \sin\left(x\right) + 27\right)}\).  Is it a power series?
    \end{hint}
    \begin{hint}
      Again, no.  The coefficient involves sine of \(x\).
    \end{hint}
    \begin{hint}
      What about \(\displaystyle\sum_{n=0}^\infty \displaystyle\frac{1}{30} \, {\left(12 \, n + 25\right)} x^{n}\)?
    \end{hint}
    \begin{hint}
      Yes, that has the correct form.  In this case \(a_n = \displaystyle\frac{2}{5} \, n + \displaystyle\frac{5}{6}\).
    \end{hint}

    \begin{multiple-choice}
      \choice[correct]{\(\displaystyle\sum_{n=0}^\infty \displaystyle\frac{1}{30} \, {\left(12 \, n + 25\right)} x^{n} \)}
      \choice{\(\displaystyle\sum_{n=0}^\infty \sin\left(x^{n}\right) \)}
      \choice{\(\displaystyle\sum_{n=0}^\infty x^{-n} \)}
      \choice{\(\displaystyle\sum_{n=0}^\infty \tan\left(x^{n}\right) \)}
      \choice{\(\displaystyle\sum_{n=0}^\infty \displaystyle\frac{2}{63} \, x^{n} {\left(14 \, \sin\left(x\right) + 27\right)} \)}
    \end{multiple-choice}
    
  \end{solution}
\end{question}

\end{document}
