\documentclass{ximera}

\title{Computing Taylor series}

\begin{document}

\begin{abstract}
  To compute the Taylor series, you need only calculate lots of derivatives.
\end{abstract}

\maketitle

There are really many perspectives from which one may approach Taylor series.  One approach goes by trying to produce ``higher order'' versions of some of the ideas and theorems from Calculus One.  For instance, in Calculus One, we had the idea of a \href{http://en.wikipedia.org/wiki/Linear_approximation}{linear approximation}, which we can improve to higher order polynomial approximations.

\youtube{https://www.youtube.com/watch?v=LJC4bpt6pBE}

You could work out these approximations around the origin\ldots

\youtube{https://www.youtube.com/watch?v=J2yoEz5bZ6I}

Or you could work out these higher-order approximations around some other point.

\youtube{https://www.youtube.com/watch?v=bEMkqb0CNLg}

You can learn how to find the Taylor series for a function by differentiating repeatedly.  Many familiar functions have nice Taylor series expansions, including trigonometric functions like sine.

\youtube{https://www.youtube.com/watch?v=eGXGlG5CuYE}

% Relevant video: taylor-series
            \begin{question}
              Now let's try it!  What are the first few terms of the Taylor series for \(f(x) = {\left(e^{x} - 1\right)} \tan\left(x\right)\) centered around the point \(a = 0\)?

              \begin{solution}
                \begin{hint}
                  
Recall that the Taylor series for \(f\) around \(a\) is given by \[\displaystyle\sum_{n=0}^\infty {f^{(n)}(a)\over n!}(x-a)^n\].
                \end{hint}
                \begin{hint}
                  In this case, we are told to consider \(a = 0\).
                \end{hint}
                \begin{hint}
                  So we must calculate derivatives of \(f\) at \(0\).
                \end{hint}
                \begin{hint}
                  Start by calculating the zeroth derivative, meaning just the function itself, at \(x = 0\).
                \end{hint}
                \begin{hint}
                  In that case, \(f(0) = 0\).
                \end{hint}
                \begin{hint}
                  Next calculate \(\displaystyle\frac{d}{dx} f(x) = {\left(\tan\left(x\right)^{2} + 1\right)} {\left(e^{x} - 1\right)} + e^{x} \tan\left(x\right)\).
                \end{hint}
                \begin{hint}
                  Therefore, \(f^{(1)}(0) = 0\).
                \end{hint}
                \begin{hint}
                  Next calculate \(f^{(2)}(x) = \displaystyle\frac{d}{dx} {\left(\tan\left(x\right)^{2} + 1\right)} {\left(e^{x} - 1\right)} + e^{x} \tan\left(x\right) = 2 \, {\left(\tan\left(x\right)^{2} + 1\right)} {\left(e^{x} - 1\right)} \tan\left(x\right) + 2 \, {\left(\tan\left(x\right)^{2} + 1\right)} e^{x} + e^{x} \tan\left(x\right)\).
                \end{hint}
                \begin{hint}
                  Therefore \(f^{(2)}(0) = 2\).
                \end{hint}
                \begin{hint}
                  This is enough to pick out the correct answer from among the given answers.
                \end{hint}
                \begin{hint}
                  Consider the \(n = 2\) term, i.e., the coefficient on \(x^2\).  Since \(f^{(2)}(0) = 2\), the coefficient must be \(1\).

                \end{hint}
                \begin{hint}
                  Looking over the possible answers, the power series for \(f\) must begin \(0 + x^{2}+ \displaystyle\frac{1}{2} \, x^{3}+ \displaystyle\frac{1}{2} \, x^{4}+ \displaystyle\frac{5}{24} \, x^{5}+ \cdots \).

                \end{hint}


              \begin{multiple-choice}
                \choice[correct]{\(0 + x^{2}+ \displaystyle\frac{1}{2} \, x^{3}+ \displaystyle\frac{1}{2} \, x^{4}+ \displaystyle\frac{5}{24} \, x^{5}+ \cdots \)}
                \choice{\(0 + 4 \, x^{2}+ \displaystyle\frac{1}{2} \, x^{3}+ \displaystyle\frac{1}{2} \, x^{4}+ \displaystyle\frac{5}{24} \, x^{5}+ \cdots \)}
                \choice{\(0 + 2 \, x^{2}+ \displaystyle\frac{1}{2} \, x^{3}+ \displaystyle\frac{1}{2} \, x^{4}+ \displaystyle\frac{5}{24} \, x^{5}+ \cdots \)}
                \choice{\(0 + \displaystyle\frac{1}{2} \, x^{3}+ \displaystyle\frac{1}{2} \, x^{4}+ \displaystyle\frac{5}{24} \, x^{5}+ \cdots \)}
                \choice{\(0 + 2 \, x^{2}+ \displaystyle\frac{1}{2} \, x^{3}+ \displaystyle\frac{1}{2} \, x^{4}+ \displaystyle\frac{5}{24} \, x^{5}+ \cdots \)}

              \end{multiple-choice}

              \end{solution}
            \end{question}

% Relevant video: series-for-sine
            \begin{question}
              Here's another example.  Consider the function \(f(x) = \cos x\).  Which of the following is the Taylor series for \(f\) around zero?
              \begin{solution}
                \begin{hint}
                  What happens when we differentiate \(f\) a bunch of times?
                \end{hint}
                \begin{hint}
                  I care because the Taylor series around zero is \(\displaystyle\sum_{n=0}^\infty \displaystyle\frac{f^{(n)}(0)}{n!} x^n\).
                \end{hint}
                \begin{hint}
                  Note that \(\displaystyle\frac{d}{dx}\cos x = -\sin x\), so \(f'(0) = 0\).
                \end{hint}
                \begin{hint}
                  Then \(\displaystyle\frac{d^2}{dx^2}\cos x = -\cos x\), so \(f^{(2)}(0) = -1\).
                \end{hint}
                \begin{hint}
                  And then \(\displaystyle\frac{d^3}{dx^3}\cos x = \sin x\), so \(f^{(3)}(0) = 0\).
                \end{hint}
                \begin{hint}
                  Finally, \(\displaystyle\frac{d^4}{dx^4}\cos x = \cos x\), so \(f^{(4)}(0) = 1\).
                \end{hint}
                \begin{hint}
                  And I say &ldquo;finally&rdquo; because now, the pattern repeats, since the fourth derivative is the same as the original function.
                \end{hint}
                \begin{hint}
                  So the derivatives of \(f\) at zero, starting from the zeroth derivative (which is just the function itself), follow the pattern: \(1,\quad 0,\quad -1,\quad 0,\quad 1,\quad 0,\quad -1,\quad 0,\quad 1,\quad 0\) and so on.
                \end{hint}
                \begin{hint}
                  Since all the odd derivatives vanish at zero, there will be no odd terms in the Taylor series for \(f\).
                \end{hint}
                \begin{hint}
                  So I could write the Taylor series as \(\displaystyle\sum_{n=0}^\infty \displaystyle\frac{f^{(2n)}(0)}{(2n)!} x^{2n}\).
                \end{hint}
                \begin{hint}
                  Since the even derivatives alternative between \(-1\) and \(1\), I can write the Taylor series as \(\displaystyle\sum_{n=0}^\infty \displaystyle\frac{(-1)^n}{(2n)!} x^{2n}\).

                \end{hint}


              \begin{multiple-choice}
                \choice[correct]{\(\displaystyle\sum_{n=0}^\infty \displaystyle\frac{(-1)^n}{(2n)!} x^{2n}\)}
                \choice{\(\displaystyle\sum_{n=0}^\infty \displaystyle\frac{(-1)^{n+1}}{(2n)!} x^{2n}\)}
                \choice{\(\displaystyle\sum_{n=0}^\infty \displaystyle\frac{(-1)^{n+1}}{(2n+1)!} x^{2n}\)}
                \choice{\(\displaystyle\sum_{n=0}^\infty \displaystyle\frac{(-1)^{n}}{(2n+1)!} x^{2n}\)}
                \choice{\(\displaystyle\sum_{n=0}^\infty \displaystyle\frac{(-1)^{n+1}}{(2n)!} x^{2n+1}\)}

              \end{multiple-choice}

              \end{solution}
            \end{question}
            

\section{What happens if we do this to a polynomial?}

% Relevant video: taylor-series-center-a
            \begin{question}
              By finding the Taylor series around \(x = -1\), rewrite the polynomial \(p(x) = -2 \, x^{3} + 4 \, x^{2} - x + 2\) as a polynomial in \((x + 1)\).

              \begin{solution}
                \begin{hint}
                  If I differentiate a polynomial enough times, I get zero.
                \end{hint}
                \begin{hint}
                  So if I compute the Taylor series for a polynomial, I just get another polynomial.
                \end{hint}
                \begin{hint}
                  That may seem pointless, but imagine writing down the Taylor series centered not at \(0\), but instead centered at \(x = -1\).
                \end{hint}
                \begin{hint}
                  Then we will have written down the same polynomial, but in terms of \((x + 1)\).
                \end{hint}
                \begin{hint}
                  Recall that the Taylor series for \(p(x)\) centered at \(-1\) is given by \[\displaystyle\sum_{n=0}^\infty \displaystyle\frac{p^{(n)}(-1)}{n!} \left(x + 1\right)^n.\]
                \end{hint}
                \begin{hint}
                  Now the zeroth derivative is just the original function, so we compute \(p^{(0)}(-1) = -2 \, \left( -1 \right)^{3} + 4 \, \left( -1 \right)^{2} - \left( -1 \right) + 2 = 9\).
                \end{hint}
                \begin{hint}
                  We differentiate to find \(p'(x) = -6 \, x^{2} + 8 \, x - 1\).
                \end{hint}
                \begin{hint}
                  Therefore \(p'(-1) = -6 \, \left( -1 \right)^{2} + 8 \, \left( -1 \right) - 1 = -15\).
                \end{hint}
                \begin{hint}
                  Differentiate again to find \(p^{(2)}(x) = -12 \, x + 8\).
                \end{hint}
                \begin{hint}
                  Therefore \(p^{(2)}(-1) = -12 \, \left( -1 \right) + 8 = 20\).
                \end{hint}
                \begin{hint}
                  Differentiate one more time to find \(p^{(3)}(x) = -12\).
                \end{hint}
                \begin{hint}
                  That is easy!  So \(p^{(3)}(-1) = -12\).
                \end{hint}
                \begin{hint}
                  Now we can put these facts into the formula for the Taylor series centered at \(-1\).
                \end{hint}
                \begin{hint}
                  We have \(p(x) = \displaystyle\frac{p^{(3)}(-1)}{6} \left(x + 1\right)^3 + \displaystyle\frac{p^{(2)}(-1)}{2} \left(x + 1\right)^2 + p^{(1)}(-1) \left(x + 1\right) + p(-1)\).
                \end{hint}
                \begin{hint}
                  Substituting what we found, we get \(p(x) = \displaystyle\frac{-12}{6} \left(x + 1\right)^3 + \displaystyle\frac{20}{2} \left(x + 1\right)^2 + -15 \left(x + 1\right) + 9\).
                \end{hint}
                \begin{hint}
                  And therefore, \(p(x) = -2 \, \left( x + 1 \right)^{3} + 10 \, \left( x + 1 \right)^{2} - 15 \, \left( x + 1 \right) + 9\).

                \end{hint}


              \begin{multiple-choice}
                \choice[correct]{\( -2 \, \left( x + 1 \right)^{3} + 10 \, \left( x + 1 \right)^{2} - 15 \, \left( x + 1 \right) + 9 \)}
                \choice{\( -2 \, \left( x + 1 \right)^{3} + 12 \, \left( x + 1 \right)^{2} - 15 \, \left( x + 1 \right) + 9 \)}
                \choice{\( -2 \, \left( x + 1 \right)^{3} + 11 \, \left( x + 1 \right)^{2} - 15 \, \left( x + 1 \right) + 9 \)}
                \choice{\( -2 \, \left( x + 1 \right)^{3} + 15 \, \left( x + 1 \right)^{2} - 15 \, \left( x + 1 \right) + 9 \)}
                \choice{\( -2 \, \left( x + 1 \right)^{3} + 6 \, \left( x + 1 \right)^{2} - 15 \, \left( x + 1 \right) + 9 \)}

              \end{multiple-choice}

              \end{solution}
            \end{question}
            

\end{document}
