\documentclass{ximera}

\title{Sequences}

\begin{document}

\begin{abstract}
  A sequence of numbers is an unending list of numbers.  Sequences form the foundation for our later discussion of series.
\end{abstract}

\maketitle

% \begin{question}
%   What is the correct answer to this question?

%   \begin{solution}
%     \begin{multiple-choice}
%       \choice[correct]{Correct answer}
%       \choice{First Distractor}
%       \choice{Second Distractor}
%       \choice{Third Distractor}
%     \end{multiple-choice}  
%   \end{solution}
% \end{question}

\begin{question}
  Consider the sequence \(a_{n}\) defined recursively by the
  rule \[a_n = {a_{n-1}} {a_{n-2}} + 3 \, {a_{n-1}} - {a_{n-2}}\] and
  the facts that \(a_0 = -3\) and \(a_1 = 5\).

  What is \(a_4\)?

  \begin{solution}
    \begin{hint}
      We have been told the first two terms of the sequence, namely \(a_0 = -3\) and \(a_1 = 5\).
    \end{hint}
    \begin{hint}
      We also have a rule \(a_n = {a_{n-1}} {a_{n-2}} + 3 \, {a_{n-1}} - {a_{n-2}}\) which lets us compute the third term \(a_2\) using these first two terms \(a_0\) and \(a_1\).
    \end{hint}
    \begin{hint}
      To compute \(a_2\), we set \(n = 2\) in the recursive rule \(a_n = {a_{n-1}} {a_{n-2}} + 3 \, {a_{n-1}} - {a_{n-2}}\).  This gives us \(a_2 = {a_{2-1}} {a_{2-2}} + 3 \, {a_{2-1}} - {a_{2-2}}\).
    \end{hint}
    \begin{hint}
      Plugging \(a_{2-1} = a_{1} = 5\) and \(a_{2-2} = a_{0} = -3\) into the rule, we learn \(a_2 = 3\).
    \end{hint}
    \begin{hint}
      To compute \(a_3\), we set \(n = 3\) in the recursive rule \(a_n = {a_{n-1}} {a_{n-2}} + 3 \, {a_{n-1}} - {a_{n-2}}\), giving \(a_3 = {a_{3-1}} {a_{3-2}} + 3 \, {a_{3-1}} - {a_{3-2}}\).
    \end{hint}
    \begin{hint}
      To evaluate that, we will have to know \(a_{3-1} = a_{2}\), but we just found out that \(a_{2} = 3\).
    \end{hint}
    \begin{hint}
      Plugging \(a_{3-1} = 3\) and \(a_{3-2} = a_{1} = 5\) into the rule, we find \(a_3 = 19\).
    \end{hint}
    \begin{hint}
      To compute \(a_4\), we set \(n = 4\) in the recursive rule \(a_n = {a_{n-1}} {a_{n-2}} + 3 \, {a_{n-1}} - {a_{n-2}}\), giving \(a_4 = {a_{4-1}} {a_{4-2}} + 3 \, {a_{4-1}} - {a_{4-2}}\).
    \end{hint}
    \begin{hint}
      To evaluate that, we will have to know \(a_{4-1} = a_{3}\), but we just found out that \(a_{3} = 19\).
    \end{hint}
    \begin{hint}
      Plugging \(a_{4-1} = 19\) and \(a_{4-2} = a_{2} = 3\) into the rule, we learn \(a_4 = 111\).
    \end{hint}
    \begin{hint}
      So we conclude \(a_4 = 111\).
    \end{hint}

    \begin{multiple-choice}
      \choice[correct]{\(111\)}
      \choice{\(176201\)}
      \choice{\(-40\)}
      \choice{\(-654016\)}
      \choice{\(2522\)}
    \end{multiple-choice}

  \end{solution}
\end{question}

What questions do you have about this topic?  What would you like to see addressed further?
\begin{free-response}
\end{free-response}

\end{document}
