\documentclass{ximera}

\title{New sequences from old}

\newcommand{\defnword}[1]{\textbf{#1}}
\newcommand{\ds}{\displaystyle}
\newcommand{\Z}{\mathbb{Z}}
\newcommand{\N}{\mathbb{N}}
\newcommand{\nth}{\mbox{\scriptsize th}}
\renewcommand{\index}[1]{}

\begin{document}

\begin{abstract}
  By throwing away some terms, we produce subsequences.
\end{abstract}

\maketitle

\label{section:new-sequences-from-old}

Given a sequence, one way to build a new sequence is to start with the
old sequence, but then throw away a whole bunch of terms.  For
instance, if we started with the sequence of perfect squares
$$
1,\quad 4,\quad 9,\quad 16,\quad 25,\quad 36,\quad 49,\quad 64,\quad 81,\quad\ldots
$$
we could throw away all the odd-indexed terms, and be left with
$$
4,\quad 16,\quad 36,\quad 64,\quad 100,\quad 144,\quad 196,\quad 256,\quad 324,\quad 400,\quad 484,\quad\ldots
$$
We say that this latter sequence is a
\defnword{subsequence}\index{sequence!subsequence}\index{subsequence}
of the original sequence.  Here is a precise definition.

\begin{definition}
  Suppose $(a_n)$ is a sequence with initial index $N$, and suppose we have a sequence of integers $(n_i)$ so that
  $$
  N \leq n_1 < n_2 < n_3 < n_4 < n_5 < \cdots 
  $$
  Then the sequence $(b_i)$ given by $b_i = a_{n_i}$ is said to be a \defnword{subsequence}\index{sequence!subsequence}\index{subsequence}
  of the sequence $a_n$.
\end{definition}

\youtube{http://www.youtube.com/watch?v=OYRRFRzf12o}

Limits are telling the story of ``what happens'' to a sequence.  If
the terms of a sequence can be made as close as desired to a limiting
value $L$, then the subsequence must share that same fate.

\begin{theorem}
  \label{theorem:subsequence-same-limit}
  If $(b_i)$ is a subsequence of the convergent sequence $(a_n)$, then
  $\ds\lim_{i \to \infty} b_i = \ds\lim_{n \to \infty} a_n$.
\end{theorem}

Of course, just because a subsequence converges does not mean that the
larger sequence converges, too.  We'll see this again in more detail,
but we'll discuss it briefly now.

\begin{example}
Find a convergent subsequence of the sequence $(a_n)$ given by the rule $a_n = (-1)^n$.
\end{example}

\begin{solution}
Note that the sequence $(a_n)$ does not converge.  But by considering the sequence of indexes $n_i = 2 \cdot i$, we can build a subsequence
$$
b_i = a_{n_i} = a_{2i} = (-1)^{2i} = 1,
$$
which is a constant sequence, so it converges to 1.
\end{solution}

There are other subsequences of $a_n = (-1)^n$ which converge but do
\textit{not} converge to one.  For instance, the subsequence of odd
indexed terms is the constant sequence $c_n = -1$, which converges to
$-1$.  For that matter, the fact that there are convergent
subsequences with distinct limits perhaps explains why the original
sequence $(a_n)$ does not converge.  Let's formalize this.

\begin{corollary}
  \label{corollary:different-subsequences-then-diverge}

  Suppose $(b_i)$ and $(c_i)$ are convergent subsequences of the sequence $(a_n)$, but
  $$
  \ds\lim_{i \to \infty} b_i \neq \ds\lim_{i \to \infty} c_i.
  $$
  Then the sequence $(a_n)$ does not converge.
\end{corollary}

\begin{proof}
  Suppose, on the contrary, the sequence $(a_n)$ did converge.  Then
  by Theorem~\xrefn{theorem:subsequence-same-limit}, the subsequence
  $(b_i)$ would converge, too, and
  $$
  \ds\lim_{i \to \infty} b_i = \ds\lim_{n \to \infty} a_n.
  $$
  Again by Theorem~\xrefn{theorem:subsequence-same-limit}, the subsequence
  $(c_i)$ would converge, too, and
  $$
  \ds\lim_{i \to \infty} c_i = \ds\lim_{n \to \infty} a_n.
  $$
  But then $\ds\lim_{i \to \infty} b_i = \ds\lim_{i \to \infty} c_i$,
  which is exactly what we are supposing doesn't happen!  To avoid
  this contradiction, it must be that our original assumption that
  $(a_n)$ converged was incorrect; in short, the sequence $(a_n)$ does
  not converge.
\end{proof}

\end{document}
