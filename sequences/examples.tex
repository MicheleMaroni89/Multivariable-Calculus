\documentclass{ximera}

\title{Examples}

\newcommand{\defnword}[1]{\textbf{#1}}
\newcommand{\ds}{\displaystyle}
\newcommand{\Z}{\mathbb{Z}}
\newcommand{\N}{\mathbb{N}}
\newcommand{\nth}{\mbox{\scriptsize th}}
\renewcommand{\index}[1]{}

\begin{document}

\begin{abstract}
  Important examples of sequences include arithmetic and geometric progressions.
\end{abstract}

\maketitle

%\marginnote{Tons of entertaining sequences are listed in the \href{http://oeis.org/}{The On-Line Encyclopedia of Integer Sequences}.}

Mathematics proceeds, in part, by finding precise statements for
everyday concepts.  We have already done this for sequences when we
found a precise definition (``function from $\N$ to $\R$'') for the
everyday concept of ``a list of real numbers.''  But all the
formalisms in the world aren't worth the paper they are printed on if
there aren't some interesting \textit{examples} of those precise
concepts.  Indeed, mathematics proceeds not only by generalizing and
formalizing, but also by focusing on specific, concrete instances.
So let me share some specific examples of sequences.

But before I can share these examples, let me address a question: how
can I hand you an example of a sequence? It is not enough just to list
off the first few terms.  Let's see why.

\begin{question}
  Consider the sequence $(a_n)$
  $$
  a_1 = 41, \quad a_2 = 43, \quad a_3 = 47, \quad a_4 = 53, \quad\ldots
  $$
  What is the next term $a_5$?  Can you identify the sequence?

  \begin{multipleChoice}
    \choice[correct]{No}
    \choice{Yes}
  \end{multipleChoice}  

  \begin{feedback}
  In spite of many so-called ``intelligence tests'' that ask questions
  just like this, this question simply doesn't have an answer.  Or
  worse, it has too many answers!
   
  This sequence might be ``the prime numbers in order, starting at
  41.''  If that's the case, then the next term is $a_5 =59$.  But
  maybe this sequence is the sequence given by the polynomial $a_n =
  n^2 - n + 41$.  If that's the case, then the next term is $a_5 =
  61$.  Who is to say which is the ``better'' answer?
  \end{feedback}
\end{question}

Now let's consider two popular ``families'' of sequences and some popular examples.


\end{document}
