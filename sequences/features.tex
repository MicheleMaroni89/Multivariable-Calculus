\documentclass{ximera}

\title{Qualitative features}

\newcommand{\defnword}[1]{\textbf{#1}}
\newcommand{\ds}{\displaystyle}
\newcommand{\Z}{\mathbb{Z}}
\newcommand{\N}{\mathbb{N}}
\newcommand{\nth}{\mbox{\scriptsize th}}
\renewcommand{\index}[1]{}

\begin{document}

\begin{abstract}
  We want some terminology with which to describe features we might
  notice about sequences.
\end{abstract}

\maketitle

\section{Qualitative features of sequences}
\label{section:qualitative-features-of-sequences}

Your first exposure to mathematics might have been about
\defnword{constructions}; you might have been asked to compute a
numeric answer or to propose a solution to a problem.  But much of
mathematics is concerned with showing \defnword{existence}, even if
the thing that is being shown to exist cannot be exhibited itself.

So sometimes we will not be able to determine the limit of a sequence,
but we still would like to know whether or not it converges to some
unspoken number.  In many cases, we can determine whether a limit
exists, without needing to---or without even being able to---compute
that limit.

\end{document}