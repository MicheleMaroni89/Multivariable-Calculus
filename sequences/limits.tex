\documentclass{ximera}

\title{Limits}

\newcommand{\defnword}[1]{\textbf{#1}}
\newcommand{\ds}{\displaystyle}
\newcommand{\Z}{\mathbb{Z}}
\newcommand{\N}{\mathbb{N}}
\newcommand{\nth}{\mbox{\scriptsize th}}
\renewcommand{\index}[1]{}

\begin{document}

\begin{abstract}
  Limits address the question ``What happens after a while?''
\end{abstract}

\maketitle

We've seen a lot of sequences, and already there are a few things we
might notice.  For instance, the arithmetic progression
$$
1,\quad 8,\quad 15,\quad 22,\quad 29,\quad 36,\quad 43,\quad 50,\quad 57,\quad 64,\quad 71,\quad 78,\quad 85,\quad 92,\quad \ldots
$$
just keeps getting bigger and bigger.  No matter how large a number
you think of, if I add enough $7$'s to $1$, eventually I will surpass
the giant number you thought of.  On the other hand, the terms in a geometric progression where each term is half the previous term, namely
$$
\frac{1}{2},\quad \frac{1}{4},\quad \frac{1}{8},\quad \frac{1}{16},\quad \frac{1}{32},\quad \frac{1}{64},\quad \frac{1}{128},\quad \frac{1}{256},\quad \frac{1}{512},\quad \frac{1}{1024},\quad \ldots ,
$$
are getting closer and closer to zero.  No matter how close you stand
near but not at zero, eventually this geometric sequence gets even closer than you
are to zero.

\youtube{https://www.youtube.com/watch?v=PRTjvMA2nCY}

These two sequences have very different stories.  One shoots off to
infinity; the other zooms in towards zero.  Mathematics is not just
about numbers; mathematics provides tools for talking about the
qualitative features of the numbers we deal with.  What about the two
sequences we just considered?  They are qualitatively very different.  The first ``goes to''
infinity; the second ``goes to'' zero.

In short, given a sequence, it is helpful to be able to say something
qualitative about it; we may want to address the question such as
``what happens after a while?'' Formally, when faced with a sequence,
we are interested in the limit
$$\lim_{i\to \infty} f(i) = \lim_{i\to\infty} a_i.$$
In Calculus One, we studied a similar question about
$$\lim_{x\to\infty} f(x)$$
when $x$ is a variable taking on real values; now, in Calculus Two, we
simply want to restrict the ``input'' values to be integers. No
significant difference is required in the definition of limit, except
that we specify, perhaps implicitly, that the variable is an integer.

\begin{definition} \relax\index{limit of a sequence}
\label{definition:limit-of-a-sequence}
Suppose that $\left(a_n\right)$ is a sequence.
To say that $\ds \lim_{n\to \infty}a_n=L$ is to say that \\
\null\quad for every $\epsilon>0$, \\
\null\quad\quad there is an $N > 0$, \\
\null\quad so that whenever $n>N$, \\
\null\quad\quad we have $|a_n-L|<\epsilon$. \\
If $\ds \lim_{n\to\infty}a_n=L$ we say that the sequence
\defnword{converges}\index{convergent
  sequence}\index{sequence!convergent}.  If there is no finite value $L$ so
that $\ds \lim_{n\to\infty}a_n = L$, then we say that the limit
\defnword{does not exist}, or equivalently that the sequence
\defnword{diverges}\index{divergent sequence}\index{sequence!divergent}.
\end{definition} 

\begin{question}
  To say that the sequence \(a_n\) converges to \(L\) means what?  In other words, what is the definition of the statement \(\displaystyle\lim_{n \to \infty} a_n = L\)?
  \begin{solution}
    \begin{hint}
      We are trying to make precise the idea that, eventually, all the terms of the sequence \(a_n\) are as close as we want to \(L\).
    \end{hint}
    \begin{hint}
      To measure closeness to \(L\), we will use a positive real number \(\epsilon\).
    \end{hint}
    \begin{hint}
      We must achieve any desired degree of closeness, so we will make a statement which is true for any positive real number \(\epsilon\).
    \end{hint}
    \begin{hint}
      In other words, the definition will begin ``For every positive real number \(\epsilon > 0\)\ldots''
    \end{hint}
    \begin{hint}
      We now must make precise the idea of ``eventually'' close.
    \end{hint}
    \begin{hint}
      We use a whole number \(N\) to capture the idea of ``sufficiently large'' values of \(n\).
    \end{hint}
    \begin{hint}
      Specifically, the definition will begin ``For every positive real number \(\epsilon > 0\), there exists an \(N \in \mathbb{N}\)\ldots''
    \end{hint}
    \begin{hint}
      The ``sufficiently large'' value of \(n\) is any value which is at least as large as \(N\).
    \end{hint}
    \begin{hint}
      So we will only consider those \(n\) for which \(n \geq N\).
    \end{hint}
    \begin{hint}
      Thus the definition goes ``For every positive real number \(\epsilon > 0\), there exists an \(N \in \mathbb{N}\) so that whenever \(n \geq N\)\ldots''
    \end{hint}
    \begin{hint}
      What happens ``eventually'' is that terms of the sequence are close to \(L\).  How close?  Within \(\epsilon\).
    \end{hint}
    \begin{hint}
      The quantity \(|a_n - L|\) is the distance between \(a_n\) and \(L\).
    \end{hint}
    \begin{hint}
      To say that \(a_n\) is within \(\epsilon\) of \(L\) is to say that \(|a_n - L| < \epsilon\).
    \end{hint}
    \begin{hint}
      Therefore the definition is ``For every positive real number \(\epsilon > 0\) there exists an \(N \in \mathbb{N}\) so that whenever \(n \geq N\), we have \( |a_n - L| < \epsilon \).''
    \end{hint}

    \begin{multiple-choice}
      \choice[correct]{For every positive real number \(\epsilon > 0\) there exists an \(N \in \mathbb{N}\) so that whenever \(n \geq N\), we have \( |a_n - L| < \epsilon \).}
      \choice{For every real number \(\epsilon > 0\) there exists an \(N \in \mathbb{N}\) so that \( |a_N - L| < \epsilon \).}
      \choice{For every real number \(\epsilon \in \mathbb{R}\) there exists an \(N \in \mathbb{N}\) so that whenever \(n \geq N\), we have \( |a_n - L| < \epsilon \).}
      \choice{For every whole number \(N > 0\) there exists a positive real number \(\epsilon > 0\) so that whenever \(n \geq N\), we have \( |a_n - L| < \epsilon \).}
      \choice{For every whole number \(N > 0\) there exists a real number \(\epsilon \in \mathbb{R}\) so that whenever \(n \geq N\), we have \( |a_n - L| < \epsilon \).}
    \end{multiple-choice}
              
  \end{solution}

  The definition of limit can be written as if it were poetry with
  line breaks and all.  Like the best of poems, it deserves to be
  memorized, performed, internalized.  Humanity struggled for millenia
  to find the wisdom contained in this definition.
\end{question}

\begin{warning}
  In the case that $\lim_{n \to \infty} a_n = \infty$, we say that
  $(a_n)$ diverges, or perhaps more precisely, we say $(a_n)$ diverges to
  infinity.  The only time we say that a sequence converges is when
  the limit exists and is equal to a \textit{finite} value.
\end{warning}

\youtube{https://www.youtube.com/watch?v=0UCRZAsIkXM}

One way to compute the limit of a sequence is to compute the limit of
a function.
\begin{theorem}
  \label{theorem:compute-limit-of-sequence-via-function}
  Let $f(x)$ be a real-valued function.  If $a_n = f(n)$ defines a
  sequence $(a_n)$ and if $\ds \lim_{x\to\infty}f(x)=L$ in the sense of Calculus
  One, then $\ds \lim_{n\to\infty} a_n=L$ as well.
\end{theorem}

\begin{example}
\label{example:find-n-for-epsilon}
Since $\ds \lim_{x\to\infty}(1/x)=0$, it is
clear that also $\ds \lim_{n\to\infty}(1/n)=0$; in other words, the sequence of numbers
$${1\over1},\quad {1\over2},\quad {1\over3},\quad {1\over4},\quad {1\over5},\quad {1\over6},\quad \ldots$$
get closer and closer to 0, or more precisely, as close as you want to get to zero, after a while, all the terms in the sequence are that close.

More precisely, no matter what $\epsilon > 0$ we pick, we can find an
$N$ big enough so that, whenever $n > N$, we have that $1/n$ is within
$\epsilon$ of the claimed limit, zero.  This can be made concrete:
let's suppose we set $\epsilon = 0.17$.  What is a suitable choice for
$N$ in response?  If we choose $N = 5$, then whenever $n > 5$ we have
$0 < 1/n < 0.17$.
\end{example}

\youtube{https://www.youtube.com/watch?v=lCW8BBBQRyc}

\begin{question}
  Consider the sequence given by the rule \[b_{n} = \displaystyle\displaystyle\frac{ 3 \, n + 9 }{ 4 \, n + 20 }.\]  For which value of \(N\) is it the case that whenever \(n \geq N\) we have that \(b_{n}\) is within \(1/50\) of \(3/4\)?

  \begin{solution}
    \begin{hint}
      There can only be one right answer.
    \end{hint}
    \begin{hint}
      So the answer is either \(N = 69\) or ``none of these.''
    \end{hint}
    \begin{hint}
      Note that \(b_{69} = \displaystyle\frac{27}{37}\).
    \end{hint}
    \begin{hint}
      Consequently, \(\left| b_{69} - \displaystyle \displaystyle\frac{3}{4} \right| = \displaystyle\frac{3}{148} \geq \displaystyle\frac{1}{50}\).
    \end{hint}
    \begin{hint}
      So the answer must be ``none of these.''
    \end{hint}

    \begin{multiple-choice}
      \choice[correct]{None of these choices for \(N\) is large enough.}
      \choice{\(N = 69 \)}
      \choice{\(N = 67 \)}
      \choice{\(N = 65 \)}
      \choice{\(N = 63 \)}
    \end{multiple-choice}

  \end{solution}
\end{question}

But it is important to note that the converse of this theorem is not
true; \label{sidenote:raining-converse}the \defnword{converse} of a
statement is what you get when you swap the assumption and the
conclusion; the converse of ``if it is raining, then it is cloudy'' is
the statement ``if it is cloudy, then it is raining.''  Which of those
statements is true?.

To show the converse is not true, it is enough to provide a single
example where it fails.  Here's the counterexample.  Recall an
instance of (a potential) general rule being broken is called a
\defnword{counterexample}.  This is a popular term among
mathematicians and philosophers.

\begin{example}
  Consider the sequence $(a_n)$ given by the rule $a_n = f(n)=\sin(n\pi)$.  This is the sequence
$$
  \sin(0\pi),\quad \sin(1\pi),\quad\sin(2\pi),\quad\sin(3\pi),\quad\ldots,
$$
which is just the sequence $0, 0, 0, 0, \ldots$ since $\sin(n\pi)=0$
whenever $n$ is an integer.  Since the sequence is just the constant sequence, we have
$$
\lim_{n\to\infty} f(n)= \lim_{n\to\infty} 0 = 0. 
$$But $\ds \lim_{x\to\infty}f(x)$, when $x$ is real, does not exist: as $x$ gets
bigger and bigger, the values $\sin(x\pi)$ do not get closer and
closer to a single value, but instead oscillate between $-1$ and $1$.
\end{example} 

Here's some general advice. If you want to know $\ds \lim_{n\to\infty}
a_n$, you might first think of a function $f(x)$ where $a_n = f(n)$,
and then attempt to compute $\ds \lim_{x\to\infty}f(x)$.  If the limit
of the function exists, then it is equal to the limit of the sequence.
But, if for some reason $\ds \lim_{x\to\infty}f(x)$ does not exist, it
may nevertheless still be the case that $\ds \lim_{n\to\infty}f(n)$
exists---you'll just have to figure out another way to compute it.


\begin{marginfigure}[0in]
\begin{tikzpicture}
	\begin{axis}[
            domain=0:20,
            ymax=2.25,
            ymin=-1.5,
            xmin=0,
            xmax=20.25,
            axis lines =middle, xlabel={$x$ and $n$}, ylabel=$y$,
            every axis y label/.style={at=(current axis.above origin),anchor=south},
            every axis x label/.style={at=(current axis.right of origin),anchor=west}
          ]
%scale = 10 ; print(join([str((n(x/scale,digits=5),n(cos(pi*(x/scale)) + (4/5)^(x/scale),digits=5))) for x in range(1,20*scale)],' '))
          \addplot [penColor, smooth] plot coordinates { (0.50000, 0.89443) (0.60000, 0.56567) (0.70000, 0.26760) (0.80000, 0.027494) (0.90000, -0.13300) (1.0000, -0.20000) (1.1000, -0.16871) (1.2000, -0.043935) (1.3000, 0.16041) (1.4000, 0.42267) (1.5000, 0.71554) (1.6000, 1.0088) (1.7000, 1.2721) (1.8000, 1.4782) (1.9000, 1.6055) (2.0000, 1.6400) (2.1000, 1.5769) (2.2000, 1.4211) (2.3000, 1.1863) (2.4000, 0.89437) (2.5000, 0.57243) (2.6000, 0.25078) (2.7000, -0.040337) (2.8000, -0.27365) (2.9000, -0.42750) (3.0000, -0.48800) (3.1000, -0.45035) (3.2000, -0.31936) (3.3000, -0.10893) (3.4000, 0.15926) (3.5000, 0.45795) (3.6000, 0.75686) (3.7000, 1.0258) (3.8000, 1.2373) (3.9000, 1.3699) (4.0000, 1.4096) (4.1000, 1.3516) (4.2000, 1.2007) (4.3000, 0.97085) (4.4000, 0.68364) (4.5000, 0.36636) (4.6000, 0.049256) (4.7000, -0.23742) (4.8000, -0.46638) (4.9000, -0.61598) (5.0000, -0.67232) (5.1000, -0.63061) (5.2000, -0.49564) (5.3000, -0.28131) (5.4000, -0.0093174) (5.5000, 0.29309) (5.6000, 0.59564) (5.7000, 0.86808) (5.8000, 1.0831) (5.9000, 1.2191) (6.0000, 1.2621) (6.1000, 1.2074) (6.2000, 1.0597) (6.3000, 0.83294) (6.4000, 0.54878) (6.5000, 0.23447) (6.6000, -0.079722) (6.7000, -0.36355) (6.8000, -0.58973) (6.9000, -0.73661) (7.0000, -0.79029) (7.1000, -0.74597) (7.2000, -0.60845) (7.3000, -0.39163) (7.4000, -0.11721) (7.5000, 0.18757) (7.6000, 0.49245) (7.7000, 0.76718) (7.8000, 0.98445) (7.9000, 1.1226) (8.0000, 1.1678) (8.1000, 1.1151) (8.2000, 0.96947) (8.3000, 0.74469) (8.4000, 0.46246) (8.5000, 0.15006) (8.6000, -0.16227) (8.7000, -0.44427) (8.8000, -0.66867) (8.9000, -0.81381) (9.0000, -0.86578) (9.1000, -0.81979) (9.2000, -0.68066) (9.3000, -0.46224) (9.4000, -0.18626) (9.5000, 0.12005) (9.6000, 0.42642) (9.7000, 0.70259) (9.8000, 0.92129) (9.9000, 1.0608) (10.000, 1.1074) (10.100, 1.0560) (10.200, 0.91171) (10.300, 0.68821) (10.400, 0.40722) (10.500, 0.096038) (10.600, -0.21510) (10.700, -0.49595) (10.800, -0.71920) (10.900, -0.86321) (11.000, -0.91410) (11.100, -0.86704) (11.200, -0.72687) (11.300, -0.50745) (11.400, -0.23045) (11.500, 0.076831) (11.600, 0.38415) (11.700, 0.66128) (11.800, 0.88087) (11.900, 1.0213) (12.000, 1.0687) (12.100, 1.0182) (12.200, 0.87474) (12.300, 0.65205) (12.400, 0.37187) (12.500, 0.061465) (12.600, -0.24891) (12.700, -0.52903) (12.800, -0.75153) (12.900, -0.89484) (13.000, -0.94502) (13.100, -0.89728) (13.200, -0.75644) (13.300, -0.53635) (13.400, -0.25874) (13.500, 0.049172) (13.600, 0.35710) (13.700, 0.63484) (13.800, 0.85501) (13.900, 0.99603) (14.000, 1.0440) (14.100, 0.99405) (14.200, 0.85108) (14.300, 0.62889) (14.400, 0.34924) (14.500, 0.039337) (14.600, -0.27055) (14.700, -0.55015) (14.800, -0.77223) (14.900, -0.91508) (15.000, -0.96482) (15.100, -0.91663) (15.200, -0.77537) (15.300, -0.55485) (15.400, -0.27684) (15.500, 0.031470) (15.600, 0.33979) (15.700, 0.61788) (15.800, 0.83845) (15.900, 0.97983) (16.000, 1.0281) (16.100, 0.97856) (16.200, 0.83594) (16.300, 0.61412) (16.400, 0.33476) (16.500, 0.025176) (16.600, -0.28440) (16.700, -0.56376) (16.800, -0.78547) (16.900, -0.92802) (17.000, -0.97748) (17.100, -0.92903) (17.200, -0.78748) (17.300, -0.56673) (17.400, -0.28842) (17.500, 0.020141) (17.600, 0.32871) (17.700, 0.60711) (17.800, 0.82785) (17.900, 0.96947) (18.000, 1.0180) (18.100, 0.96867) (18.200, 0.82625) (18.300, 0.60463) (18.400, 0.32549) (18.500, 0.016113) (18.600, -0.29326) (18.700, -0.57244) (18.800, -0.79395) (18.900, -0.93632) (19.000, -0.98559) (19.100, -0.93695) (19.200, -0.79523) (19.300, -0.57430) (19.400, -0.29584) (19.500, 0.012890) (19.600, 0.32162) (19.700, 0.60019) (19.800, 0.82107) (19.900, 0.96285)};
          \node at (axis cs:2.2000, 1.4211) [anchor=south west] {\color{penColor}$f(x)$};  
          \node at (axis cs:10, 1.15) [anchor=south] {\color{penColor2}$a_n$};
% print(join([str((x,n(cos(pi*x) + (4/5)^x,digits=5))) for x in range(1,20)],' ')) 
          \addplot[color=penColor2,fill=penColor2,only marks,mark=*] coordinates{(1, -0.20000) (2, 1.6400) (3, -0.48800) (4, 1.4096) (5, -0.67232) (6, 1.2621) (7, -0.79029) (8, 1.1678) (9, -0.86578) (10, 1.1074) (11, -0.91410) (12, 1.0687) (13, -0.94502) (14, 1.0440) (15, -0.96482) (16, 1.0281) (17, -0.97748) (18, 1.0180) (19, -0.98559) (20, 1.0115)};

        \end{axis}
\end{tikzpicture}
\caption{Plots of $f(x) = \cos (\pi \, x) + (4/5)^x$ and the sequence $a_n = (-1)^n + (4/5)^n$.}
\label{fig:graphs-of-sequences}
\end{marginfigure}

\end{document}