\documentclass{ximera}

\title{Geometric progressions}

\newcommand{\defnword}[1]{\textbf{#1}}
\newcommand{\ds}{\displaystyle}
\newcommand{\Z}{\mathbb{Z}}
\newcommand{\N}{\mathbb{N}}
\newcommand{\nth}{\mbox{\scriptsize th}}
\renewcommand{\index}[1]{}

\begin{document}

\begin{abstract}
  In a geometric progression, the ratio between subsequent terms is
  a fixed quantity.
\end{abstract}

\maketitle

The second family we consider are geometric progressions.

\youtube{https://www.youtube.com/watch?v=1z8QKFFU3Hc}

\begin{definition}
  A \defnword{geometric progression} (sometimes called a geometric
  sequence)\index{geometric progression} is a sequence where the ratio
  between subsequent terms is the same, fixed quantity.
\end{definition}

\begin{example}
  An example of a geometric progression is the sequence $(a_n)$ starting
  $$
  a_1 = 10, \quad a_2 = 30, \quad a_3 = 90, \quad a_4 = 270, \quad\ldots
  $$
  and given by the rule $a_n = 10 \cdot 3^{n-1}$.  Each term is three
  times the preceding term.
\end{example}

In general, a geometric progression in which the ratio between
subsequent terms is $r$ can be written as
$$
a_n = a_1 \cdot r^{n-1}.
$$
Alternatively, we could describe a geometric progression
recursively, by giving a starting value $a_1$, and using the rule that
$a_{n} = r \cdot a_{n-1}$.

\begin{remark}
Why are geometric progressions called \textit{geometric?}  Note that every term is the \defnword{geometric mean} of its two neighbors.  The geometric mean of two numbers $a$ and $b$ is defined to be $\sqrt{ab}$.

Of course, that raises another question: why is the geometric mean called \textit{geometric?}  One geometric interpretation of the geometric mean of $a$ and $b$ is this: the geometric mean is the side length of a square whose area is equal to that of the rectangle having side lengths $a$ and $b$.
\end{remark}

A geometric progression needn't be increasing.  For instance, in the following geometric progression
$$
\frac{7}{5}, \quad \frac{7}{10}, \quad \frac{7}{20}, \quad \frac{7}{40}, \quad \frac{7}{80}, \quad \frac{7}{160}, \quad\ldots
$$
the ratio between subsequent terms is one half, and each term is smaller than the previous.

\begin{question}
  Which of the following could be the initial terms of a geometric progression?
              
  \begin{solution}
    \begin{hint}
      In a geometric progression, there is a common ratio between any two neighboring terms.
    \end{hint}
    \begin{hint}
      For instance, the ratio between \(-64\) and \(16\) is \(-4\).
    \end{hint}
    \begin{hint}
      Can \(1,  -4,  16,  -48,  192,  -768 \) be the beginning of a geometric progression?  No, because the ratio between most of those terms is \(-4\), but \(16\) and \(-48\) break the pattern: the ratio \(\displaystyle\displaystyle\frac{-48}{16}\) is -3, not -4.
    \end{hint}
    \begin{hint}
      Can \(1,  -4,  16,  -64,  320,  -1280 \) be the beginning of a geometric progression?  No, because the ratio between most of those terms is \(-4\), but \(-64\) and \(320\) break the pattern: the ratio \(\displaystyle\displaystyle\frac{320}{-64}\) is -5, not -4.
    \end{hint}
    \begin{hint}
      Can \(1,  -4,  16,  -64,  256,  -1280 \) be the beginning of a geometric progression?  No, because the ratio between most of those terms is \(-4\), but \(256\) and \(-1280\) break the pattern: the ratio \(\displaystyle\displaystyle\frac{-1280}{256}\) is -5, not -4.
    \end{hint}
    \begin{hint}
      Can \(1,  -4,  16,  -64,  256,  -1024 \) be the beginning of a geometric progression?  Yes, because the ratio between each neighboring pair of terms is \(-4\).  For example, 
      \(\displaystyle\displaystyle\frac{16}{-4} = \displaystyle\displaystyle\frac{-64}{16} = \displaystyle\displaystyle\frac{256}{-64} = -4\).
    \end{hint}

    \begin{multiple-choice}
      \choice[correct]{\(1,\quad -4,\quad 16,\quad -64,\quad 256,\quad -1024,\quad\ldots \)}
      \choice{\(1,\quad -5,\quad 20,\quad -80,\quad 320,\quad -1280,\quad\ldots \)}
      \choice{\(1,\quad -4,\quad 20,\quad -80,\quad 320,\quad -1280,\quad\ldots \)}
      \choice{\(1,\quad -4,\quad 16,\quad -48,\quad 192,\quad -768,\quad\ldots \)}
      \choice{\(1,\quad -4,\quad 16,\quad -64,\quad 320,\quad -1280,\quad\ldots \)}
    \end{multiple-choice}

  \end{solution}
\end{question}


\end{document}
