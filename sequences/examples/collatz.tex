\documentclass{ximera}

\title{Collatz sequence}

\newcommand{\defnword}[1]{\textbf{#1}}
\newcommand{\ds}{\displaystyle}
\newcommand{\Z}{\mathbb{Z}}
\newcommand{\N}{\mathbb{N}}
\newcommand{\nth}{\mbox{\scriptsize th}}
\renewcommand{\index}[1]{}

\begin{document}

\begin{abstract}
  There remain many unanswered questions about the Collatz sequence.
\end{abstract}

\maketitle

Here is a fun sequence with which to amuse your friends---or distract
your enemies.  Let's start our sequence with $a_1 = 6$.  Subsequent
terms are defined using the rule
$$
a_n = \begin{cases} a_{n-1} / 2 & \mbox{ if $a_{n-1}$ is even, and } \\
3 \, a_{n-1} + 1 & \mbox{ if $a_{n-1}$ is odd.}
\end{cases}
$$
Let's compute $a_2$.  Since $a_1$ is even, we follow the instructions
in the first line, to find that $a_2 = a_1/2 = 3$. To compute $a_3$,
note that $a_2$ is odd so we follow the instruction in the second
line, and $a_3 = 3 \, a_2 + 1 = 3 \cdot 3 + 1 = 10$.  Since $a_3$ is
even, the first line applies, and $a_4 = a_3 / 2 = 10 / 2 = 5$.  But
$a_4$ is odd, so the second line applies, and we find $a_5 = 3 \cdot 5
+ 1 = 16$.  And $a_5$ is even, so $a_6 = 16 / 2 = 8$.  And $a_6$ is
even, so $a_7 = 8/4 = 4$.  And $a_7$ is even, so $a_8 = 4 / 2 = 2$,
and then $a_9 = 2/2 = 1$.  Oh, but $a_9$ is odd, so $a_{10} = 3 \cdot
1 + 1 = 4$.  And it repeats.  Let's write down the start of this sequence:
$$
6,\quad %1 
3,\quad %2
10,\quad  %3
5,\quad  %4
16,\quad  %5
8,\quad  %6
4,\quad  %7
2,\quad  %8
1,\quad  %9
4,\quad %10
2,\quad %11
1,\quad %12
\overbrace{4,\quad %10
2,\quad %11
1,}^{\mbox{repeats}}\quad %12
4,\quad %10
\ldots
$$
What if we had started with a number other than six?  What if we set
$a_1 = 25$ but then we used the same rule?  In that case, since $a_1$
is odd, we compute $a_2$ by finding $3 \, a_1 + 1 = 3 \cdot 25 + 1 =
76$.  Since $76$ is even, the next term is half that, meaning $a_3 =
38$.  If we keep this up, we find that our sequence begins
\begin{align*}
&25,\quad 76,\quad 38,\quad 19,\quad 58,\quad 29,\quad 88,\quad 44,\quad 22,\quad 11,\quad 34,\quad 17,\quad 52,\quad 26, \\
&13,\quad 40,\quad 20,\quad 10,\quad 5,\quad 16,\quad 8,\quad 4,\quad 2, \quad 1, \quad \ldots
\end{align*}
and then it repeats ``4, 2, 1, 4, 2, 1, \ldots'' just like before.

If you think you have an argument that answers the Collatz conjecture, I challenge you to try your hand at the $5x+1$ conjecture, that is, use the rule
\[
a_n = \displaystyle\begin{cases} a_{n-1} / 2 & \mbox{ if $a_{n-1}$ is even, and } \\
5 \, a_{n-1} + 1 & \mbox{ if $a_{n-1}$ is odd.}
\end{cases}
\]

Does this always happen?  Is it true that no matter which positive
integer you start with, if you apply the half-if-even, $3x+1$-if-odd
rule, you end up getting stuck in the ``4, 2, 1, \ldots'' loop?  That
this is true is the \defnword{Collatz conjecture}; it has been
verified for all starting values below $5 \times 2^{60}$.  Nobody has
found a value which doesn't return to one, but for all anybody knows
there \textit{might} well be a very large initial value which doesn't
return to one; nobody knows either way.  It is an unsolved
problem\sidenote{This is not the last unsolved problem we will
  encounter in this course.  There are many things which humans do not
  understand.} in mathematics.

\end{document}
