\documentclass{ximera}

\title{Arithmetic progressions}

\newcommand{\defnword}[1]{\textbf{#1}}
\newcommand{\ds}{\displaystyle}
\newcommand{\Z}{\mathbb{Z}}
\newcommand{\N}{\mathbb{N}}
\newcommand{\nth}{\mbox{\scriptsize th}}
\renewcommand{\index}[1]{}

\begin{document}

\begin{abstract}
  In a arithmetic progression, each term differs from the next by the same quantity.
\end{abstract}

\maketitle

%\marginnote{Tons of entertaining sequences are listed in the \href{http://oeis.org/}{The On-Line Encyclopedia of Integer Sequences}.}

The first family we consider are the ``arithmetic'' sequences.  Here
is a definition.

\youtube{https://www.youtube.com/watch?v=WOF5kkVekUY}

%{Mathematically, the word \defnword{family}
%  does not have an entirely precise definition; a family of things is
%  a \defnword{collection} or a \defnword{set} of things, but family
%  also has a connotation of some sort of relatedness.} 

\begin{definition}
  An \defnword{arithmetic progression} (sometimes called an arithmetic
  sequence)\index{arithmetic progression} is a sequence where each
  term differs from the next by the same, fixed quantity.
\end{definition}

\begin{example}
  An example of an arithmetic progression is the sequence $(a_n)$ which begins 
  $$
  a_1 = 10, \quad a_2 = 14, \quad a_3 = 18, \quad a_4 = 22, \quad\ldots
  $$
  and which is given by the rule $a_n = 6 + 4 \, n$.  Each term differs
  from the previous by four.
\end{example}

In general, an arithmetic progression in which subsequent terms differ
by $m$ can be written as
$$
a_n = m \, (n-1) + a_1.
$$
Alternatively, we could describe an arithmetic progression
recursively, by giving a starting value $a_1$, and using the rule that
$a_{n} = a_{n-1} + m$.

%\marginnote{Why are arithmetic progressions called \textit{arithmetic?}  Note that every term is the \defnword{arithmetic mean}, that is, the \defnword{average}, of its two neighbors.}

An arithmetic progression can decrease; for instance,
$$
17,\quad  15,\quad  13,\quad  11,\quad  9, \quad\ldots
$$
is an arithmetic progression.

\begin{question}
  Which of the following could be the initial terms of an arithmetic progression?

  \begin{solution}
    \begin{hint}
      In an arithmetic progression, there is a common difference between any two neighboring terms.
    \end{hint}
    \begin{hint}
      For instance, the difference between \(-3\) and \(-1\) is \(-2\).
    \end{hint}
    \begin{hint}
      Can \(3,  1,  -1,  -4,  -6,  -8 \) be the beginning of an arithmetic progression?  No, because the difference most between of those terms is \(-2\), but \(-1\) and \(-4\) break the pattern, with difference \(-3\), not \(-2\).
    \end{hint}
    \begin{hint}
      Can \(3,  1,  -1,  -3,  -4,  -6 \) be the beginning of an arithmetic progression?  No, because the difference between most of those terms is \(-2\), but \(-3\) and \(-4\) break the pattern, with difference \(-1\), not \(-2\).
    \end{hint}
    \begin{hint}
      Can \(3,  1,  -1,  -3,  -5,  -8 \) be the beginning of an arithmetic progression?  No, because the difference between most of those terms is \(-2\), but \(-5\) and \(-8\) break the pattern, with difference \(-3\), not \(-2\).
    \end{hint}
    \begin{hint}
      Can \(3,  1,  -1,  -3,  -5,  -7 \) be the beginning of an arithmetic progression?  Yes, because the difference between each neighboring pair of terms is \(-2\).  For example, 
      \(\left(-1\right) - \left(1\right) = \left(-3\right) - \left(-1\right) = \left(-5\right) - \left(-3\right) = -2\).
    \end{hint}

    \begin{multiple-choice}
      \choice[correct]{\(3,\quad 1,\quad -1,\quad -3,\quad -5,\quad -7,\quad\ldots \)}
      \choice{\(3,\quad 0,\quad -2,\quad -4,\quad -6,\quad -8,\quad\ldots \)}
      \choice{\(3,\quad 1,\quad 0,\quad -2,\quad -4,\quad -6,\quad\ldots \)}
      \choice{\(3,\quad 1,\quad -1,\quad -4,\quad -6,\quad -8,\quad\ldots \)}
      \choice{\(3,\quad 1,\quad -1,\quad -3,\quad -4,\quad -6,\quad\ldots \)}
    \end{multiple-choice}

  \end{solution}
\end{question}

\end{document}
