\documentclass{ximera}

\title{Monotonicity}

\newcommand{\defnword}[1]{\textbf{#1}}
\newcommand{\ds}{\displaystyle}
\newcommand{\Z}{\mathbb{Z}}
\newcommand{\N}{\mathbb{N}}
\newcommand{\nth}{\mbox{\scriptsize th}}
\renewcommand{\index}[1]{}
\renewcommand{\label}[1]{}

\begin{document}

\begin{abstract}
  A monotone sequence keeps heading in the same direction.
\end{abstract}

\maketitle

\label{subsection:monotonicity}

And sometimes we don't even care about limits, but we'd simply like
some terminology with which to describe features we might notice about
sequences.  Here is some of that terminology.

(For instance, how much money I have on day $n$ is a sequence; I
probably hope that sequence is an increasing sequence!)

\begin{definition}
  A sequence is called
  \defnword{increasing}\index{sequence!increasing} (or sometimes
  \defnword{strictly increasing}) if $\ds a_n<a_{n+1}$ for all $n$.
  It is called {\dfont
    non-decreasing\index{sequence!non-decreasing}\/} if $\ds a_n\le
  a_{n+1}$ for all $n$.

  Similarly a sequence is {\dfont
    decreasing\index{sequence!decreasing}\/} (or, by some people,
  \defnword{strictly decreasing}) if $\ds a_n>a_{n+1}$ for all $n$ and
  {\dfont non-increasing\index{sequence!non-increasing}\/} if $\ds
  a_n\ge a_{n+1}$ for all $n$.
\end{definition}
To make matters worse, the people who insist on saying ``strictly
increasing'' may---much to everybody's confusion---insist on calling a
non-decreasing sequence ``increasing.'' I'm not going to play their
game; I'll be careful to say ``non-decreasing'' when I mean a sequence
which is getting larger or staying the same.

To make matters better, lots of facts are true for sequences which are
either increasing or decreasing; to talk about this situation without
constantly saying ``either increasing or decreasing,'' we can make up
a single word to cover both cases.
\begin{definition}
  If a sequence is increasing, non-decreasing, decreasing, or
  non-increasing, it is said to be {\dfont
    monotonic\index{sequence!monotonic}\/}.
\end{definition}

\youtube{https://www.youtube.com/watch?v=FMaKP0hmytU}

Let's see some examples of sequences which are monotonic.
\begin{example}
The sequence $\ds a_n = {2^n-1\over2^n}$ which starts
$$
  {1\over2},\quad {3\over4},\quad {7\over8},\quad {15\over16},\quad \ldots,
$$
is increasing.  On the other hand, the sequence $\ds b_n = {n+1\over n}$, which starts
$$ 
  {2\over1},\quad{3\over2},\quad{4\over3},\quad{5\over4},\quad\ldots,
$$
is decreasing.
\end{example}

\begin{question}
  Which of the following could be the initial terms of a monotonic sequence?

  \begin{solution}
    \begin{hint}
      A monotonic sequence is a sequence which is either increasing, decreasing, non-increasing, or non-decreasing.
    \end{hint}
    \begin{hint}
      Looking at available choices, mostly these sequences are increasing.
    \end{hint}
    \begin{hint}
      For example, \(12 < 13\).
    \end{hint}
    \begin{hint}
      Can \(5,  11,  16,  12,  18,  22 \) be the beginning of a monotonic sequence?  No, because mostly these terms are increasing, but \(16\) and \(12\) break the pattern.
    \end{hint}
    \begin{hint}
      Can \(5,  11,  16,  22,  16,  22 \) be the beginning of a monotonic sequence?  No, because mostly these terms are increasing, but \(22\) and \(16\) break the pattern.
    \end{hint}
    \begin{hint}
      Can \(5,  6,  10,  12,  15,  10 \) be the beginning of a monotonic sequence?  No, because mostly these terms are increasing, but \(15\) and \(10\) break the pattern.
    \end{hint}
    \begin{hint}
      Can \(5,  11,  12,  13,  15,  16 \) possibly be the beginning of a monotonic sequence?  Yes, because this portion of the sequence is increasing, so it is possible that the rest of sequence continues that pattern.
    \end{hint}


    \begin{multiple-choice}
      \choice[correct]{\(5,\quad 11,\quad 12,\quad 13,\quad 15,\quad 16,\quad\ldots \)}
      \choice{\(5,\quad 3,\quad 6,\quad 8,\quad 12,\quad 15,\quad\ldots \)}
      \choice{\(5,\quad 10,\quad 4,\quad 10,\quad 15,\quad 19,\quad\ldots \)}
      \choice{\(5,\quad 11,\quad 16,\quad 12,\quad 18,\quad 22,\quad\ldots \)}
      \choice{\(5,\quad 11,\quad 16,\quad 22,\quad 16,\quad 22,\quad\ldots \)}
      
    \end{multiple-choice}
    
  \end{solution}
\end{question}
            

\end{document}
