\documentclass{ximera}

\title{Boundedness}

\newcommand{\defnword}[1]{\textbf{#1}}
\newcommand{\ds}{\displaystyle}
\newcommand{\Z}{\mathbb{Z}}
\newcommand{\N}{\mathbb{N}}
\newcommand{\nth}{\mbox{\scriptsize th}}
\renewcommand{\index}[1]{}
\renewcommand{\label}[1]{}

\begin{document}

\begin{abstract}
  Limits address the question ``What happens after a while?''
\end{abstract}

\maketitle

Sometimes we can't say exactly which number a sequence approaches, but
we can at least say that the sequence doesn't get too big or too
small.

\begin{definition}
  \label{definition:sequence-bounded}
  A sequence $(a_n)$ is \defnword{bounded
    above}\index{sequence!bounded above} if there is some number
$M$ so that for all $n$, we have $\ds a_n\le M$.  Likewise, a sequence
$(a_n)$ is {\dfont bounded below\index{sequence!bounded below}\/} if
there is some number $M$ so that for every $n$, we have $\ds a_n\ge M$.

If a sequence is both bounded above and bounded below, the sequence is said
to be {\dfont bounded\index{sequence!bounded}\/}.
\end{definition}

\youtube{https://www.youtube.com/watch?v=FC4TCSk-O24}

\begin{question}
  To say that the sequence \(a_n\) is ``bounded below'' is to say what?
  \begin{solution}
    \begin{hint}
      The definition begins by asserting the existence of some bound \(M\).
    \end{hint}
    \begin{hint}
      The bound is a real number, meaning \(M \in \mathbb{R}\).
    \end{hint}
    \begin{hint}
      So the definition begins ``there exists an \(M \in \mathbb{R}\)\ldots''
    \end{hint}
    \begin{hint}
      The bound must hold for all terms in the sequence.
    \end{hint}
    \begin{hint}
      So we will assert something for all indexes \(n \in \mathbb{N}\).
    \end{hint}
    \begin{hint}
      The definition begins ``there exists an \(M \in \mathbb{R}\), so that for all \(n \in \mathbb{N}\), \ldots''
    \end{hint}
    \begin{hint}
      For a particular term \(a_n\) to be bounded below by \(M\) just means that \(a_n \geq M\).
    \end{hint}
    \begin{hint}
      Altogether then, being bounded below means ``there exists an \(M \in \mathbb{R}\), so that for all \(n \in \mathbb{N}\), we have \(a_n \geq M\).''  Sometimes you might see this ending with \( a_n > M \), but that is a difference which does not affect which sequences are bounded below.
      
    \end{hint}
    
    \begin{multiple-choice}
      \choice[correct]{There exists an \(M \in \mathbb{R}\), so that for all \(n \in \mathbb{N}\), we have \(a_n \geq M\).}
      \choice{There exists an \(n \in \mathbb{N}\), so that for all \(M \in \mathbb{R}\), we have \(a_n \geq M\).}
      \choice{For all \(M \in \mathbb{R}\), there exists an \(n \in \mathbb{N}\), so that \(a_n \geq M\).}
      \choice{For all \(n \in \mathbb{N}\), there exists an \(M \in \mathbb{R}\), so that \(a_n \geq M\).}
      \choice{There exists an \(M \in \mathbb{R}\), so that for all \(n \in \mathbb{N}\), we have \(a_n \leq M\).}
    \end{multiple-choice}

  \end{solution}
\end{question}
            
If a sequence $\ds
\{a_n\}_{n=0}^\infty$ is increasing or non-decreasing it is bounded
below (by $\ds a_0$), and if it is decreasing or non-increasing it is
bounded above (by $\ds a_0$).

\begin{question}
  Consider the sequence \(b_{n} = -n^{2} + 5 \, n - 5\).  Is the sequence bounded above?  Bounded below?
  \begin{solution}
    \begin{hint}
      Consider the coefficient on \(n^{2}\) in \(b_{n} = -n^{2} + 5 \, n - 5\), which is \(-1\).
    \end{hint}
    \begin{hint}
      Since the leading term's coefficient is negative, when \(n\) is large, \(b_{n}\) is very negative.
    \end{hint}
    \begin{hint}
      Consequently, the sequence is not bounded below.
    \end{hint}
    \begin{hint}
      At least for \(n\) large, the sequence is decreasing.
    \end{hint}
    \begin{hint}
      Consequently, the sequence is bounded above.
    \end{hint}
    \begin{hint}
      Altogether then, the sequence is bounded above, but not below.
    \end{hint}

    \begin{multiple-choice}
      \choice[correct]{Bounded above, but not bounded below.}
      \choice{Bounded below, but not bounded above.}
      \choice{Bounded above and bounded below.}
      \choice{Bounded neither above nor below.}
    \end{multiple-choice}

  \end{solution}
\end{question}
            

\end{document}
