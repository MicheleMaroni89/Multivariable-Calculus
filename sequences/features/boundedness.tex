\documentclass{ximera}

\title{Boundedness}

\newcommand{\defnword}[1]{\textbf{#1}}
\newcommand{\ds}{\displaystyle}
\newcommand{\Z}{\mathbb{Z}}
\newcommand{\N}{\mathbb{N}}
\newcommand{\nth}{\mbox{\scriptsize th}}
\renewcommand{\index}[1]{}

\begin{document}

\begin{abstract}
  Limits address the question ``What happens after a while?''
\end{abstract}

\maketitle

\label{subsection:boundedness}

Sometimes we can't say exactly which number a sequence approaches, but
we can at least say that the sequence doesn't get too big or too
small.

\begin{definition}
  \label{definition:sequence-bounded}
  A sequence $(a_n)$ is \defnword{bounded
    above}\index{sequence!bounded above} if there is some number
$M$ so that for all $n$, we have $\ds a_n\le M$.  Likewise, a sequence
$(a_n)$ is {\dfont bounded below\index{sequence!bounded below}\/} if
there is some number $M$ so that for every $n$, we have $\ds a_n\ge M$.

If a sequence is both bounded above and bounded below, the sequence is said
to be {\dfont bounded\index{sequence!bounded}\/}.
\end{definition}

\youtube{https://www.youtube.com/watch?v=FC4TCSk-O24}

\begin{question}
  To say that the sequence \(a_n\) is ``bounded below'' is to say what?
  \begin{solution}
    \begin{hint}
      The definition begins by asserting the existence of some bound \(M\).
    \end{hint}
    \begin{hint}
      The bound is a real number, meaning \(M \in \mathbb{R}\).
    \end{hint}
    \begin{hint}
      So the definition begins ``there exists an \(M \in \mathbb{R}\)\ldots''
    \end{hint}
    \begin{hint}
      The bound must hold for all terms in the sequence.
    \end{hint}
    \begin{hint}
      So we will assert something for all indexes \(n \in \mathbb{N}\).
    \end{hint}
    \begin{hint}
      The definition begins ``there exists an \(M \in \mathbb{R}\), so that for all \(n \in \mathbb{N}\), \ldots''
    \end{hint}
    \begin{hint}
      For a particular term \(a_n\) to be bounded below by \(M\) just means that \(a_n \geq M\).
    \end{hint}
    \begin{hint}
      Altogether then, being bounded below means ``there exists an \(M \in \mathbb{R}\), so that for all \(n \in \mathbb{N}\), we have \(a_n \geq M\).''  Sometimes you might see this ending with \( a_n > M \), but that is a difference which does not affect which sequences are bounded below.
      
    \end{hint}
    
    \begin{multiple-choice}
      \choice[correct]{There exists an \(M \in \mathbb{R}\), so that for all \(n \in \mathbb{N}\), we have \(a_n \geq M\).}
      \choice{There exists an \(n \in \mathbb{N}\), so that for all \(M \in \mathbb{R}\), we have \(a_n \geq M\).}
      \choice{For all \(M \in \mathbb{R}\), there exists an \(n \in \mathbb{N}\), so that \(a_n \geq M\).}
      \choice{For all \(n \in \mathbb{N}\), there exists an \(M \in \mathbb{R}\), so that \(a_n \geq M\).}
      \choice{There exists an \(M \in \mathbb{R}\), so that for all \(n \in \mathbb{N}\), we have \(a_n \leq M\).}
    \end{multiple-choice}

              \end{solution}
            \end{question}
            

If a sequence $\ds
\{a_n\}_{n=0}^\infty$ is increasing or non-decreasing it is bounded
below (by $\ds a_0$), and if it is decreasing or non-increasing it is
bounded above (by $\ds a_0$).

% Relevant video: sequence-bounded
\begin{question}
  Consider the sequence \(b_{n} = -n^{2} + 5 \, n - 5\).  Is the sequence bounded above?  Bounded below?
  \begin{solution}
    \begin{hint}
      Consider the coefficient on \(n^{2}\) in \(b_{n} = -n^{2} + 5 \, n - 5\), which is \(-1\).
    \end{hint}
    \begin{hint}
      Since the leading term's coefficient is negative, when \(n\) is large, \(b_{n}\) is very negative.
    \end{hint}
    \begin{hint}
      Consequently, the sequence is not bounded below.
    \end{hint}
    \begin{hint}
      At least for \(n\) large, the sequence is decreasing.
    \end{hint}
    \begin{hint}
      Consequently, the sequence is bounded above.
    \end{hint}
    \begin{hint}
      Altogether then, the sequence is bounded above, but not below.
    \end{hint}


    \begin{multiple-choice}
      \choice[correct]{Bounded above, but not bounded below.}
      \choice{Bounded below, but not bounded above.}
      \choice{Bounded above and bounded below.}
      \choice{Bounded neither above nor below.}
    \end{multiple-choice}

  \end{solution}
\end{question}
            





Finally, with all this new terminology we can state the most important
theorem of Chapter~\xrefn{chapter:sequences}.

\begin{theorem} If the sequence $a_n$ is bounded and monotonic, then
  $\lim_{n \to \infty} a_n$ exists.  \label{thm:bounded-monotonic}
\end{theorem}
In short, bounded monotonic sequences converge---though we can't
necessarily describe the number to which they converge.

We will not prove this theorem in the textbook.\sidenote{Proving this
  theorem is, honestly, the purview of a course in \textit{analysis},
  the theoretical underpinnings of calculus.  That's not to say it
  couldn't be done in this course, but I intend this to be a ``first
  glance'' at sequences---so much will be left unsaid.} Nevertheless,
it is not hard to believe: suppose that a sequence is increasing and
bounded, so each term is larger than the one before, yet never larger
than some fixed value $M$. The terms must then get closer and closer
to some value between $\ds a_0$ and $M$. It certainly need not be $M$,
since $M$ may be a ``too generous'' upper bound; the limit will be the
smallest number that is above\sidenote{This concept of the ``smallest
  number above all the terms'' is an incredibly important one; it is
  the idea of a
  \href{http://en.wikipedia.org/wiki/Least-upper-bound_property}{least
    upper bound} that underlies the real numbers.} all of the terms
$\ds a_n$.  Let's try an example!

\begin{example}
  \label{example:sequence-bounded}
  All of the terms $\ds (2^i-1)/2^i$ are less than 2, and the sequence
  is increasing.  As we have seen, the limit of the sequence is 1---1
  is the smallest number that is bigger than all the terms in the
  sequence.  Similarly, all of the terms $(n+1)/n$ are bigger than
  $1/2$, and the limit is 1---1 is the largest number that is smaller
  than the terms of the sequence.
\end{example}

We don't actually need to know that a sequence is monotonic to apply
this theorem---it is enough to know that the sequence is
``eventually'' monotonic,\sidenote{After all, the limit only depends on
  what is happening after some large index, so throwing away the
  beginning of a sequence won't affect its convergence or its limit.}
that is, that at some point it becomes increasing or decreasing.  For
example, the sequence $10$, $9$, $8$, $15$, $3$, $21$, $4$, $3/4$,
$7/8$, $15/16$, $31/32,\ldots$ is not increasing, because among the
first few terms it is not. But starting with the term $3/4$ it is
increasing, so if the pattern continues and the sequence is bounded,
the theorem tells us that the ``tail'' $3/4, 7/8, 15/16, 31/32,\ldots$
converges.  Since convergence depends only on what happens as $n$ gets
large, adding a few terms at the beginning can't turn a convergent
sequence into a divergent one.

\begin{example}
\label{example:nth-root-of-n}
Show that the sequence $(a_n)$ given by $a_n = n^{1/n}$ converges.
\end{example}

\marginnote{You may be worried about my saying that $\log 3 > 1$.  If
  $\log$ were the common (base~10) logarithm, this would be wrong, but
  as far as I'm concerned, there is only one log, the natural log.
  Since $3 > e$, we may conclude that $\log 3 > 1$.}

\begin{solution}
  We might first show that this sequence is decreasing, that is, we show
  that for all $n$,
  $$
  n^{1/n} > (n+1)^{1/(n+1)}.
  $$
  But this isn't true!  Take a look
  \begin{align*}
    a_1 &= 1, \\
    a_2 &= \sqrt{2} \approx 1.4142, \\
    a_3 &= \sqrt[3]{3} \approx 1.4422, \\
    a_4 &= \sqrt[4]{4} \approx 1.4142, \\
    a_5 &= \sqrt[5]{5} \approx 1.3797, \\
    a_6 &= \sqrt[6]{6} \approx 1.3480, \\
    a_7 &= \sqrt[7]{7} \approx 1.3205, \\
    a_8 &= \sqrt[8]{8} \approx 1.2968, \mbox{ and}\\
    a_9 &= \sqrt[9]{9} \approx 1.2765. \\
  \end{align*}
  But it does seem that this sequence perhaps is decreasing after the
  first few terms.  Can we justify this?

  Yes!  Consider the real function $\ds f(x)=x^{1/x}$ when $x\ge1$.
  We compute the derivative---perhaps via ``logarithmic differentiation''---to find
  $$
  f'(x)=\frac{x^{1/x} \, (1-\log x)}{x^2}.
  $$
  Note that when $x\ge 3$, the derivative $f'(x)$ is negative.  Since the function $f$ is decreasing, we can conclude that the sequence is decreasing---well, at least for $n \geq 3$.

  Since all terms of the sequence are positive, the sequence is
  decreasing and bounded when $n \ge 3$, and so the sequence converges.
\end{solution}

\marginnote{As it happens, you could compute the limit in
  Example~\xrefn{example:nth-root-of-n}, but our given solution shows that
  it converges even without knowing the limit!}

\begin{example}
Show that the sequence $a_n = \ds\frac{n!}{n^n}$ converges.
\end{example}

\begin{solution}
  Let's get an idea of what is going on by computing the first few terms.
% print(join(['a_' + str(k) + '=' + latex(f(x=k)) + ' ' + '\\approx ' + str(n(f(x=k),digits=5)) for k in range(1,9)],',\quad '))
\begin{align*}
a_1&= 1,\quad a_2= \frac{1}{2},\quad a_3= \frac{2}{9} \approx 0.22222,\quad a_4= \frac{3}{32} \approx 0.093750, \\
a_5&= \frac{24}{625} \approx 0.038400, \quad a_6= \frac{5}{324} \approx 0.015432, \\
a_7&= \frac{720}{117649} \approx 0.0061199,\quad a_8= \frac{315}{131072} \approx 0.0024033.
\end{align*}
  The sequence appears to be decreasing.  To formally show this, we would need to show $\ds a_{n+1}< a_n$, but we will instead show that
$$
\frac{a_{n+1}}{a_n} < 1,
$$
which amounts to the same thing.  It is helpful trick here to think of
the ratio between subsequent terms, since the factorials end up
canceling nicely.  In particular,
\begin{align*}
  {a_{n+1}\over a_n} &= {(n+1)!\over (n+1)^{n+1}}{n^n\over n!} \\
  &= {(n+1)!\over n!}{n^n\over (n+1)^{n+1}} \\
  &= {n+1\over n+1}\left({n\over n+1}\right)^n
  &= \left({n\over n+1}\right)^n < 1.
\end{align*}
  Note that the sequence is bounded below, since every term is positive.

  Because the sequence is decreasing and bounded below, it converges.
  Indeed, Exercise~\xrefn{exercise:factorial-limit} asks you to
  compute the limit.
\end{solution}

These sorts of arguments involving the ratio of subsequent terms will
come up again in a big way in Section~\xrefn{section:ratio-test}.
Stay tuned!

%%%%%%%%%%%%%%%%%%%%%%%%%%%%%%%%%%%%%%%%%%%%%%%%%%%%%%%%%%%%%%%%
\begin{exercises}

\begin{exercise} \label{exercise:exponential limit}
Compute $\ds\lim_{x\to\infty} x^{1/x}$.
\begin{answer} $1$
\end{answer}\end{exercise}

\begin{exercise} Use the squeeze theorem to show that 
$\ds\lim_{n\to\infty} {n!\over n^n}=0$.
\label{exercise:factorial-limit}
\end{exercise}

\begin{exercise} Determine whether $\ds\{\sqrt{n+47}-\sqrt{n}\}_{n=0}^\infty$ 
converges or diverges. If it converges, compute the limit.
\begin{answer} $0$
\end{answer}\end{exercise}

\begin{exercise} Determine whether 
$\ds\left\{{n^2+1\over (n+1)^2}\right\}_{n=0}^\infty$ 
converges or diverges. If it converges, compute the limit.
\begin{answer} $1$
\end{answer}\end{exercise}

\begin{exercise} Determine whether 
$\ds\left\{{n+47\over\sqrt{n^2+3n}}\right\}_{n=1}^\infty$ 
converges or diverges. If it converges, compute the limit.
\begin{answer} $1$
\end{answer}\end{exercise}

\begin{exercise} Determine whether 
$\ds\left\{{2^n\over n!}\right\}_{n=0}^\infty$ 
converges or diverges. If it converges, compute the limit.
\begin{answer} $0$
\end{answer}\end{exercise}

\end{exercises}



\end{document}