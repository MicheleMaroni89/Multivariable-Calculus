\documentclass{ximera}

\title{Defining sequences}

\newcommand{\defnword}[1]{\textbf{#1}}
\newcommand{\ds}{\displaystyle}
\newcommand{\Z}{\mathbb{Z}}
\newcommand{\N}{\mathbb{N}}
\newcommand{\nth}{\mbox{th}}
\renewcommand{\index}[1]{}

\begin{document}

\begin{abstract}
  Sequences can be defined via a rule or recursively.
\end{abstract}

\maketitle

\subsection{Defining sequences by giving a rule}

Just as real-valued functions were usually expressed by a formula, we
will most often encounter sequences that can be expressed by a
formula.  In the Introduction to this textbook, we saw the sequence
given by the rule $\ds a_i=f(i)=1-1/2^i$.  Other examples are easy to
cook up, like
\begin{align*}
  a_i &={i\over i+1}, \\
  b_n &={1\over2^n}, \\
  c_n &=\sin(n\pi/6), \mbox{ or} \\
  d_i &={(i-1)(i+2)\over2^i}. \\
\end{align*}
Frequently these formulas will make sense if thought of either as
functions with domain $\R$ or $\N$, though occasionally the given
formula will make sense only for integers.  We'll address the
idea of a real-valued function ``filling in'' the gaps between the
terms of a sequence when we look at graphs in
Section~\xrefn{section:graphs}.

\begin{warning}
  A common misconception is to confuse the sequence with the rule for
  generating the sequence.  The sequences $(a_n)$ and $(b_n)$ given by
  the rules $a_n = (-1)^n$ and $b_n = \cos (\pi \, n)$ are, despite
  appearances, different rules which give rise to the \textit{same}
  sequence.  These are just different names for the same object.
\end{warning}

Let's give a precise definition for ``the same'' when speaking of
sequences.  Compare this to equality for functions: two functions are
the same if they have same domain and codomain, and they assign the
same value to each point in the domain.

\youtube{https://www.youtube.com/watch?v=5f5-d3uiYxo}

\begin{definition}
  Suppose $(a_n)$ and $(b_n)$ are sequences starting at $1$.  These
  sequences are \defnword{equal}\index{sequence!equality} if for all
  natural numbers $n$, we have $a_n = b_n$.

  More generally, two sequences $(a_n)$ and $(b_n)$ are
  \defnword{equal} if they have the same initial index $N$, and for
  every integer $n \geq N$, the $n^{\nth}$ terms have the same value, that is,
  \[
  a_n = b_n \quad \mbox{for all $n \geq N$.}
  \]
\end{definition}
In other words, sequences are the same if they have the same set of
valid indexes, and produce the same real numbers for each of those
indexes---regardless of whether the given ``rules'' or procedures for
computing those sequences resemble each other in any way.

\subsection{Defining sequences using previous terms}
\label{subsection:recursive-definition}

Another way to define a sequence is \textit{recursively}, that is, by
defining the later outputs in terms of previous outputs.  We start by
defining the first few terms of the sequence, and then describe how
later terms are computed in terms of previous terms.

\begin{example}
Define a sequence recursively by
$$
a_1 = 1, \quad a_2 = 3, \quad a_3 = 10,
$$
and the rule that $a_n = a_{n-1} - a_{n-3}$.  Compute $a_5$.
\end{example}

\begin{worked-solution}
  First we compute $a_4$.  Substituting $4$ for $n$ in the rule $a_n = a_{n-1} - a_{n-3}$, we find
$$
a_4 = a_{4-1} - a_{4-3} = a_3 - a_1.
$$
But we have values for $a_3$ and $a_1$, namely $10$ and $1$, respectively.  Therefore $a_4 = 10 - 1 = 9$.

Now we are in a position to compute $a_5$.  Substituting $5$ for $n$ in the rule $a_n = a_{n-1} - a_{n-3}$, we find
$$
a_5 = a_{5-1} - a_{5-3} = a_4 - a_2.
$$
We just computed $a_4 = 9$; we were given $a_2 = 3$.  Therefore $a_5 = 9 - 3 = 6$.
\end{worked-solution}

\youtube{http://www.youtube.com/watch?v=Krqi7UGJV5o}

\begin{question}
  Consider the sequence \(a_{n}\) defined recursively by the
  rule \[a_n = {a_{n-1}} {a_{n-2}} + 3 \, {a_{n-1}} - {a_{n-2}}\] and
  the facts that \(a_0 = -3\) and \(a_1 = 5\).  What is \(a_4\)?

  \begin{solution}
    \begin{hint}
      We have been told the first two terms of the sequence, namely \(a_0 = -3\) and \(a_1 = 5\).
    \end{hint}
    \begin{hint}
      We also have a rule \(a_n = {a_{n-1}} {a_{n-2}} + 3 \, {a_{n-1}} - {a_{n-2}}\) which lets us compute the third term \(a_2\) using these first two terms \(a_0\) and \(a_1\).
    \end{hint}
    \begin{hint}
      To compute \(a_2\), we set \(n = 2\) in the recursive rule \(a_n = {a_{n-1}} {a_{n-2}} + 3 \, {a_{n-1}} - {a_{n-2}}\).  This gives us \(a_2 = {a_{2-1}} {a_{2-2}} + 3 \, {a_{2-1}} - {a_{2-2}}\).
    \end{hint}
    \begin{hint}
      Plugging \(a_{2-1} = a_{1} = 5\) and \(a_{2-2} = a_{0} = -3\) into the rule, we learn \(a_2 = 3\).
    \end{hint}
    \begin{hint}
      To compute \(a_3\), we set \(n = 3\) in the recursive rule \(a_n = {a_{n-1}} {a_{n-2}} + 3 \, {a_{n-1}} - {a_{n-2}}\), giving \(a_3 = {a_{3-1}} {a_{3-2}} + 3 \, {a_{3-1}} - {a_{3-2}}\).
    \end{hint}
    \begin{hint}
      To evaluate that, we will have to know \(a_{3-1} = a_{2}\), but we just found out that \(a_{2} = 3\).
    \end{hint}
    \begin{hint}
      Plugging \(a_{3-1} = 3\) and \(a_{3-2} = a_{1} = 5\) into the rule, we find \(a_3 = 19\).
    \end{hint}
    \begin{hint}
      To compute \(a_4\), we set \(n = 4\) in the recursive rule \(a_n = {a_{n-1}} {a_{n-2}} + 3 \, {a_{n-1}} - {a_{n-2}}\), giving \(a_4 = {a_{4-1}} {a_{4-2}} + 3 \, {a_{4-1}} - {a_{4-2}}\).
    \end{hint}
    \begin{hint}
      To evaluate that, we will have to know \(a_{4-1} = a_{3}\), but we just found out that \(a_{3} = 19\).
    \end{hint}
    \begin{hint}
      Plugging \(a_{4-1} = 19\) and \(a_{4-2} = a_{2} = 3\) into the rule, we learn \(a_4 = 111\).
    \end{hint}
    \begin{hint}
      So we conclude \(a_4 = 111\).
    \end{hint}

    \begin{multiple-choice}
      \choice[correct]{\(111\)}
      \choice{\(176201\)}
      \choice{\(-40\)}
      \choice{\(-654016\)}
      \choice{\(2522\)}
    \end{multiple-choice}

  \end{solution}
\end{question}

\end{document}


%%% Local Variables: 
%%% mode: latex
%%% TeX-master: t
%%% End: 
