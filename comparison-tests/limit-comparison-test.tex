\documentclass{ximera}

\title{Limit comparison test}

\begin{document}

\begin{abstract}
  The limit comparison test relates convergence to how quickly the terms are converging to zero.
\end{abstract}

\maketitle

\section{What is the limit comparison test?}

Sometimes the comparison test can be sort of tricky to apply; it may be intuitively obvious that ``eventually'' a series will resemble a convergent \(p\)-series, but actually setting up an inequality between two series may be arithmetically challenging.  The limit comparison test is a way around this.  Specifically, if two sequences \((a_n)\) and \((b_n)\) of positive numbers have the property that \(\lim_{n\to\infty} a_n/b_n = L > 0\) for some real number \(L\), then the series \(sum_{n=1}^\infty a_n\) and \(\sum_{n=1}^\infty b_n\) either both converge or both diverge.  It is often easy to check that \(\lim_{n\to\infty} a_n/b_n = L > 0\) so this turns out to be a useful test.

\youtube{https://www.youtube.com/watch?v=dtXg4jMZG4M}

% Relevant video: limit-comparison-test
\begin{question}
  Does the series \(\displaystyle\sum_{n=3}^\infty \left( \displaystyle\frac{2 \, n^{2} + 5 \, n + 6}{6 \, n^{4} + n^{3} + 4 \, n^{2} + 1} \right)\) converge or diverge?
  
  \begin{solution}
    \begin{hint}
      Consider how the largest power in the numerator compares to the largest power in the denominator.
    \end{hint}
    \begin{hint}
      The largest power in the numerator is 2.
    \end{hint}
    \begin{hint}
      The largest power in the denominator is 4.
    \end{hint}
    \begin{hint}
      We will apply the limit comparison test.
    \end{hint}
    \begin{hint}
      Set \(a_{n} = \displaystyle\frac{2 \, n^{2} + 5 \, n + 6}{6 \, n^{4} + n^{3} + 4 \, n^{2} + 1}\).
    \end{hint}
    \begin{hint}
      Set \(b_{n} = \displaystyle\frac{1}{n^{2}}\).
    \end{hint}
    \begin{hint}
      Consider \(\lim_{n\to \infty} a_n/b_n\).
    \end{hint}
    \begin{hint}
      In this case, \(\lim_{n\to \infty} a_n/b_n = \displaystyle\frac{2 \, n^{2} + 5 \, n + 6}{6 \, n^{4} + n^{3} + 4 \, n^{2} + 1} \cdot \displaystyle\frac{n^{2}}{1}\).
    \end{hint}
    \begin{hint}
      And so, \(\lim_{n\to \infty} a_n/b_n = L = \displaystyle\frac{1}{3}\).
    \end{hint}
    \begin{hint}
      Because \(L = \displaystyle\frac{1}{3} > 0\), we may conclude that \(\displaystyle\sum_{n=3}^\infty a_{n}\) and \(\displaystyle\sum_{n=3}^\infty b_{n}\) either both converge or both diverge.
    \end{hint}
    \begin{hint}
      The series \(\displaystyle\sum_{n=3}^\infty b_{n}\) converges by \(p\)-series test.
    \end{hint}
    \begin{hint}
      So the series \(\displaystyle\sum_{n=3}^\infty a_{n}\) converges, too, by the Limit Comparison Test.
      
    \end{hint}
    
    \begin{multiple-choice}
      \choice[correct]{The series converges.}
      \choice{The series diverges.}
      
    \end{multiple-choice}
    
  \end{solution}
\end{question}

One fun thing about the limit comparison test is that it justifies our thinking coarsely about convergence, which is not just practically useful, but also provides us deeper intuition about just what convergence really means.

\youtube{https://www.youtube.com/watch?v=xMI3kpYqNN8}


\end{document}
