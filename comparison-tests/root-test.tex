\documentclass{ximera}

\title{Root test}

\begin{document}

\begin{abstract}
  The root test begins by considering the limit of the n-th root of the n-th term.
\end{abstract}

\maketitle

\section{What is the root test?}

The root test is somewhat similar to the ratio test.  Again, suppose \(a_n > 0\) and you want to determine whether the series \(\sum_{n=1}^\infty a_n\) converges or diverges.  Compute \(L = \displaystyle\lim_{n\to\infty} \sqrt[n]{a_{n}}\), and if \(L &lt; 1\), the series converges, while if \(L > 1\), the series diverges.  If \(L = 1\), then the test is inconclusive.

It is no secret that I do not think too highly of the ratio test.

\yotube{https://www.youtube.com/watch?v=a5TGqoge7hA}

Nevertheless, there are situations where the root test proves the convergence of a series for which the ratio test is silent.

% Relevant video: root-test-useless
\begin{question}
  Does the series \(\displaystyle\sum_{n=7}^\infty \left( \left(\displaystyle\frac{5 \, n + 4}{7 \, n + 8}\right)^{n} \right)\) converge or diverge?

  \begin{solution}
    \begin{hint}
      Let's set \(a_n = \left(\displaystyle\frac{5 \, n + 4}{7 \, n + 8}\right)^{n}\).
    \end{hint}
    \begin{hint}
      This series involves powers of \(n\), so the ratio test might be helpful.
    \end{hint}
    \begin{hint}
      So \(\displaystyle\displaystyle\frac{a_{n+1}}{a_n} = \displaystyle\frac{\left(\displaystyle\frac{5 \, n + 9}{7 \, n + 15}\right)^{n + 1}}{\left(\displaystyle\frac{5 \, n + 4}{7 \, n + 8}\right)^{n}}\).
    \end{hint}
    \begin{hint}
      That does not look like a particularly fruitful path to go down; perhaps we could try the root test?
    \end{hint}
    \begin{hint}
      Note that \(\sqrt[n]{a_{n}} = \displaystyle\frac{5 \, n + 4}{7 \, n + 8}\).
    \end{hint}
    \begin{hint}
      We calculate \(\lim_{n \to \infty} \displaystyle\frac{5 \, n + 4}{7 \, n + 8} = \displaystyle\frac{5}{7}\).
    \end{hint}
    \begin{hint}
      So \(L = \displaystyle\frac{5}{7}\).
    \end{hint}
    \begin{hint}
      Since \(L < 1\), we may conclude by the root test that the given series coverges.
    \end{hint}


    \begin{multiple-choice}
      \choice[correct]{The series converges.}
      \choice{The series diverges.}
    \end{multiple-choice}
  \end{solution}

\end{question}
            


<h4>What are \(p\)-series?</h4>

<p>A very important example of a series is a \(p\)-series, which just means a series of the form \[\sum_{n=1}^\infty \frac{1}{n^p}\] for some fixed \(p\).  We can show that <%= linkto_video('p-series','a \(p\)-series converges when \(p > 1\) and diverges otherwise') %>.  
This is <%= linkto_textbook('thm:p-series') %>, but you can also find a discussion using condensation in <%= linkto_textbook('subsection:p-series') %>.</p>

<img style="float: right; margin-left: 12pt;" class="img-thumbnail" src="https://d396qusza40orc.cloudfront.net/sequence%2Fimages%2Flogo.png">

<h4>How large can the overhang in a stack of blocks be?</h4>

<p>One of the more unusual &ldquo;applications&rdquo; of the harmonic series is <%= linkto_video('harmonic-tower','using it to build towers of blocks') %>; this is the same object that appears in our course logo.  I also made a <a href="http://www.youtube.com/watch?v=I3ilNAcjNng"><i class="icon-facetime-video"></i>&nbsp;video of the block tower on YouTube</a> if you want to see it from a different angle.</p>

<div style="clear: both;"></div>

<hr/>

<h4>By the end of the week&hellip;</h4>

<p>I have some learning objectives for you.  At the end of this week, I hope you will be able to</p>
<ul>
<li>apply the ratio test on a <%= linkto_video('ratio-test-example','series involving powers' )%>,</li>
<li>apply the ratio test on a <%= linkto_video('ratio-test-statement','series involving factorials')%>,</li>
<li>apply the ratio test on a <%= linkto_video('ratio-test-one-over-e','series involving both factorials and powers') %>,</li>
<li>identify a <%= linkto_video('p-series','\(p\)-series') %>,</li>
<li>compare a <%= linkto_video('reciprocal-n-log-n','series to a \(p\)-series') %>,</li>
<li>apply the <%= linkto_video('root-test-useless','root test') %> to analyze convergence, and</li>
<li>bound the value of a series by using the <%= linkto_video('integral-test','integral test') %>.</li>
</ul>
<p>When you feel ready, you can <a href="/sequence-001/quiz/attempt?quiz_id=77"><i class="icon-pencil"></i>&nbsp;work on the homework</a>.  The homework is &ldquo;formative assessment&rdquo; meaning I intend the homework as a teaching tool, not to determine your &ldquo;score&rdquo; in the course.  So please, use the hints when you get stuck.  Discuss freely on the forums.  Do the homework again and again&mdash;the problems are randomly generated.  Feel free to use the provided resources in whatever way helps you to understand the material.  You can master this material!</p>

\end{document}
