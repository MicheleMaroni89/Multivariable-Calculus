\documentclass{ximera}

\title{Series}

\begin{document}

\begin{abstract}
  A series is, roughly speaking, the sum of a sequence of numbers.
\end{abstract}

\maketitle

%%%%%%%%%%%%%%%%%%%%%%%%%%%%%%%%%%%%%%%%%%%%%%%%%%%%%%%%%%%%%%%%
\section{What does $\displaystyle\sum a_k = L$ mean?}
\youtube{https://www.youtube.com/watch?v=2i8XQrjCai4}

\begin{question}
  To say that \(\displaystyle\displaystyle\sum_{k=4}^\infty a_k = L\) means what?  In other words, what does it mean to say that the ``value'' of a series is \(L\)?
  \begin{solution}
    \begin{hint}
      We are trying to make precise the idea that, by adding up sufficiently many of the terms of the sequence \(a_k\), we can get as close to \(L\) as desired.
    \end{hint}
    \begin{hint}
      We consider the sequence of partial sums, \((s_n)\).
    \end{hint}
    \begin{hint}
      For this course, the term \(s_n\) is the sum of the terms through index \(n\).  Be warned though some people might define it differently: in some texts, the term \(s_n\) may be the sum of the first \(n\) terms.
    \end{hint}
    \begin{hint}
      So \(s_n = \displaystyle\displaystyle\sum_{k=4}^n a_k\).
    \end{hint}
    \begin{hint}
      To say that we can get as close as desired to \(L\) by adding up many terms is to say that \(s_n\) is as close as desired to \(L\) whenever \(n\) is large enough.
    \end{hint}
    \begin{hint}
      This is saying that the sequence \(s_n\) converges to \(L\).
    \end{hint}
    \begin{hint}
      So to say that \(\displaystyle\displaystyle\sum_{k=4}^\infty a_k = L\) means that the sequence of partial sums \(s_n = \displaystyle\displaystyle\sum_{k=4}^n a_k \) converges to \(L\).
      
    \end{hint}
    
    
    \begin{multiple-choice}
      \choice[correct]{The sequence of partial sums \(s_n = \displaystyle\displaystyle\sum_{k=4}^n a_k \) converges to \(L\).}
      \choice{The sequence of partial sums \(s_n = \displaystyle\displaystyle\sum_{k=n}^\infty a_k \) converges to \(L\).}
      \choice{The sequence of partial sums \(s_n = \displaystyle\displaystyle\sum_{k=4}^n a_n \) converges to \(L\).}
      \choice{The sequence of partial sums \(s_n = \displaystyle\displaystyle\sum_{k=1}^n a_k \) converges to \(L\).}
      \choice{The sequence of partial sums \(s_n = \displaystyle\displaystyle\sum_{k=1}^n a_k \) is bounded by \(L\).}
      
    \end{multiple-choice}
    
  \end{solution}
\end{question}

\hrule

%%%%%%%%%%%%%%%%%%%%%%%%%%%%%%%%%%%%%%%%%%%%%%%%%%%%%%%%%%%%%%%%
\section{Geometric series}

What does $\displaystyle\sum_{k=0}^\infty (1/2)^k$ equal?

\youtube{https://www.youtube.com/watch?v=d9ewMEXRqmY}

What is the value of $\sum_{k=0}^\infty r^k$?

\youtube{https://www.youtube.com/watch?v=bcvAGpm0ANY}

What is the value of $\displaystyle\sum_{k=m}^\infty r^k$?

\youtube{https://www.youtube.com/watch?v=IU6ShMXtSe8}

\begin{question}
  Evaluate \(\displaystyle\sum_{n=2}^\infty \left(\displaystyle\frac{1}{2}\right)^{n}\).

  \begin{solution}
    \begin{hint}
      A basic formula to evaluate a geometric series is \(\displaystyle\sum_{n=0}^\infty r^{n} = \displaystyle\frac{1}{1-r}\).
    \end{hint}
    \begin{hint}
      But this series does not start with \(n=0\), so we multiply both sides by \(r^{2}\) to get  \(\displaystyle\sum_{n=2}^\infty r^{n} = \displaystyle\frac{r^{2}}{1-r}\).
    \end{hint}
    \begin{hint}
      In this case, \(r = \displaystyle\frac{1}{2}\), so \(\displaystyle\sum_{n=2}^\infty \left( \displaystyle\frac{1}{2} \right)^{n} = \displaystyle\frac{\left( \displaystyle\frac{1}{2} \right)^{2}}{1-\left(\displaystyle\frac{1}{2}\right)}\).
    \end{hint}
    \begin{hint}
      Evaluating this expression, we find \(\displaystyle\sum_{n=2}^\infty \left( \displaystyle\frac{1}{2} \right)^{n} = \displaystyle\frac{1}{2}\).
    \end{hint}


    \begin{multiple-choice}
      \choice[correct]{\(\displaystyle\frac{1}{2}\)}
      \choice{\(-\displaystyle\frac{1}{65610}\)}
      \choice{\(\displaystyle\frac{1}{6}\)}
      \choice{\(\displaystyle\frac{1}{768}\)}
      \choice{\(\displaystyle\frac{1}{16}\)}
      
    \end{multiple-choice}
    
  \end{solution}
\end{question}

\hrule

\section{Telescoping series}

What is $\displaystyle\sum_{k=1}^\infty \frac{1}{(k+1)k}$?

\youtube{https://www.youtube.com/watch?v=f2vnyzCNX3M}

\begin{question}
  Evaluate \(\displaystyle\sum_{n=3}^\infty \displaystyle\frac{25}{25 \, n^{2} + 45 \, n + 14}\).

  \begin{solution}
    \begin{hint}
      As the title says, this is a telescoping series.
    \end{hint}
    \begin{hint}
      So we should try to rewrite \(\displaystyle\frac{25}{25 \, n^{2} + 45 \, n + 14}\) as \(a_{n} - a_{n+1}\).
    \end{hint}
    \begin{hint}
      The denominator \(25 \, n^{2} + 45 \, n + 14\) factors as \({\left(5 \, n + 7\right)} {\left(5 \, n + 2\right)}\).
    \end{hint}
    \begin{hint}
      So we can write \(\displaystyle\frac{25}{25 \, n^{2} + 45 \, n + 14}\) as \(a_{n} - a_{n+1}\) where \(a_{n} = \displaystyle\frac{5}{5 \, n + 2}\).
    \end{hint}
    \begin{hint}
      Therefore \(\displaystyle\sum_{n=3}^\infty \displaystyle\frac{25}{25 \, n^{2} + 45 \, n + 14} = \displaystyle\sum_{n=3}^\infty \left( a_{n} - a_{n+1} \right)\).
    \end{hint}
    \begin{hint}
      This is a telescoping series, so \(\displaystyle\sum_{n=3}^\infty \displaystyle\frac{25}{25 \, n^{2} + 45 \, n + 14} = a_{3} - \lim_{n \to \infty} a_{n}\).
    \end{hint}
    \begin{hint}
      Note that \(\lim_{n \to \infty} \displaystyle\frac{5}{5 \, n + 2} = 0\), so \(\lim_{n \to \infty} a_{n} = 0\).
    \end{hint}
    \begin{hint}
      So \(\displaystyle\sum_{n=3}^\infty \displaystyle\frac{25}{25 \, n^{2} + 45 \, n + 14} = a_{3}\).
    \end{hint}
    \begin{hint}
      Specifically then, \(\displaystyle\sum_{n=3}^\infty \displaystyle\frac{25}{25 \, n^{2} + 45 \, n + 14} = \displaystyle\frac{5}{17}\).
    \end{hint}
    
    \begin{multiple-choice}
      \choice[correct]{\(\displaystyle\frac{5}{17}\)}
      \choice{\(1\)}
      \choice{\(\displaystyle\frac{1}{10}\)}
      \choice{\(\displaystyle\frac{1}{8}\)}
      \choice{\(\displaystyle\frac{1}{7}\)}
    \end{multiple-choice}
    
  \end{solution}
\end{question}
            


\end{document}
